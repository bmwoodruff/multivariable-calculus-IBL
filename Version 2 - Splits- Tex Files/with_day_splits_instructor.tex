
\documentclass[letterpaper,oneside]{book}%

%Several If statements. These control various comments:
%  True/False -> if true is uncommented, these are INCLUDED.
% instructor -> notes for instructor
% notes -> Footnotes
% thomas -> Problems/References to Thomas's Calculus 11th edition
% larson -> Problem/References to Larson's Calculus, 5th edition
% stewart -> Problems/References to Stewart's Calculus: Early Transcendentals, 7th edition
% bmw -> Comments & Hints from Ben Woodruff
% valpo -> Comments specifically for Valparaiso University Students

\newif\ifinstructor
\instructortrue
%\instructorfalse

\newif\ifnotes
\notestrue
%\notesfalse

\newif\ifthomas
%\thomastrue
\thomasfalse

\newif\iflarson
%\larsontrue
\larsonfalse

\newif\ifstewart
\stewarttrue
%\stewartfalse

\newif\ifbmw
%\bmwtrue
\bmwfalse

\newif\ifvalpo
\valpotrue
%\valpofalse


\usepackage[left=1in,right=2.75in,top=1in,bottom=1in]{geometry}
\marginparwidth 1.75in

\usepackage{tabls}
\usepackage{booktabs}
\usepackage{amsmath}
\usepackage{amssymb}
\usepackage{amsthm}
\usepackage{amsfonts}
\usepackage{multicol}
\usepackage{enumitem}
\usepackage{microtype}
\usepackage{tikz}
\usetikzlibrary{positioning}
\usepackage{multirow}
\usepackage{comment}
\usepackage{graphicx}
\usepackage{wrapfig}
\usepackage{color}
\definecolor{darkblue}{rgb}{0, 0, .6}
\usepackage{hyperref}
\hypersetup{
	colorlinks=true,
	linkcolor=darkblue,
	anchorcolor=darkblue,
	citecolor=darkblue,
	urlcolor=darkblue,
}

\usepackage{minitoc}

\newcommand{\wrapup}{
\bmw{\section{Wrap Up}
Once you have finished the problems in the section and feel comfortable with the ideas, create a short one page lesson plan that contains examples of the key ideas.  You will get a chance to teach from this lesson plan prior to taking the exam. Then log on to Brainhoney and download the quiz. Once you have taken the quiz, you can upload your work back to brainhoney and then download the key to see how you did. If you still need to work on mastering some of the ideas, please do so and then demonstrate your mastery though the quiz corrections.}

%I liked Ben's wrapup for his Diff-Eq book better. So I just stole it and modified a little!
\valpo{\section{Wrap Up}
This concludes the chapter.  Look at the objectives at the beginning of the chapter. Can you now do all the things you were promised? \\
\textbf{Lesson Plan Creation\\}
Your assignment: organize what you've learned into a small collection of examples that illustrates the key concepts. I'll call this your one-page lesson plan. You may use both sides. The objectives at the beginning of the chapter give you a list of the key concepts. Once you finish your lesson plan, scan it into a PDF document (use any scanner on campus), and then upload the document to Blackboard.

As you create this lesson plan, consider the following:
\begin{itemize}
 \item On the class period after making this plan, you'll have 20 minutes in class where you will get to teach a peer your examples. If you keep the examples simple, you'll be able to fully review the entire chapter.
 \item Before each Celebration of Knowledge \instructor{This is just an exam} we will devote a class period to review. With well created lesson plans, you will have 4-8 pages(for 2-4 Chapters) to review for each, instead of 50-100 problems.
 \item Think ahead 2-5 years. If you make these lesson plans correctly, you'll be able to look back at your lesson plans for this semester. In about 10 pages, you can have the entire course summarized and easy for you to recall.
\end{itemize}
} %endvalpo
} %end wrapup

\newcommand{\myscale}{1}
\newcommand{\ds}{\displaystyle}
\newcommand{\dfdx}[1]{\frac{d#1}{dx}}
\newcommand{\ddx}{\frac{d}{dx}}


\let\oldmarginpar\marginpar
\renewcommand\marginpar[1]{\-\oldmarginpar{\raggedright\footnotesize #1}}
%\renewcommand\marginpar[1]{\-\oldmarginpar[\raggedleft\footnotesize #1]{\raggedright\footnotesize #1}}


%\usepackage[12hr]{datetime}
%\newdateformat{draftdate}{%
%\shortdayofweekname{\THEDAY}{\THEMONTH}{\THEYEAR}, %
%\THEDAY\ \shortmonthname[\THEMONTH] \THEYEAR}
%\draftdate
%\usepackage{eso-pic}
%\AddToShipoutPicture{\put(10,10){\small Draft: \today\ at \currenttime }}%--- version: \MakeUppercase{\svnInfoRevision}}}

%Note Both of the instructor and 'notes' appear in colors (blue, red respectively) to make them stand out more in the digital copy.

% Instructor-specific material (answers, helps, etc.)
\ifinstructor
  \newcommand{\instructor}[1]{\marginpar{\textcolor{blue}{\textbf{Instructor: }#1}}}
\else
  \newcommand{\instructor}[1]{}
\fi

%These are notes between authors of the text, primarily Ben Woodruff and Jason Grout
\ifnotes
\renewcommand{\thefootnote}{\roman{footnote}}
\newcommand{\note}[1]{\footnote{#1}\marginpar{\fbox{\textbf{\thefootnote}}}}
\else
\newcommand{\note}[1]{}
\fi

%Specific comments, problems, etc relevant to Ben Woodruff's Class
\ifbmw
	\newcommand{\bmw}[1]{#1}
	\newcommand{\marginparbmw}[1]{\bmw{\marginpar{#1}}}
\else
	\newcommand{\bmw}[1]{}
	\newcommand{\marginparbmw}[1]{}
\fi
	
%Notes for Students @ Valparaiso University
\ifvalpo
	\newcommand{\valpo}[1]{\textbf{To Valpo Students:} #1}
\else
	\newcommand{\valpo}[1]{}
\fi	

%The next three (or more) are adding refences to specific, copyrighted textbooks for additional problems.
% Notes for Thomas, 11th edition
\ifthomas
	\newcommand{\thomasee}[1]{Thomas: #1}
\else
	\newcommand{\thomasee}[1]{}
\fi

%Notes for Larson, 5th Edition
\iflarson
	\newcommand{\larsonfive}[1]{Larson: #1}
\else
	\newcommand{\larsonfive}[1]{}
\fi

%Notes for Stewart, 7th Edition
\ifstewart
	\newcommand{\stewarts}[1]{Stewart: #1}
\else
	\newcommand{\stewarts}[1]{}
\fi



\theoremstyle{plain}
\newtheorem{theorem}{Theorem}[chapter]
\newtheorem*{theorem*}{Theorem}
\newtheorem{lemma}[theorem]{Lemma}
\newtheorem*{lemma*}{Lemma}
\newtheorem{proposition}[theorem]{Proposition}
\newtheorem{corollary}[theorem]{Corollary}

\renewcommand{\chaptername}{Unit}
\setcounter{chapter}{-1}
\newcounter{unitday}[chapter]
\newcommand{\uday}{\newpage \LARGE \center  Day \theunitday \normalsize \flushleft \stepcounter{unitday} \vskip0.1in}


\newtheoremstyle{box}%
{}{}% standard spacing before and after
{}% Body style
{}{\bfseries}{.}% Heading indent, font, and punctuation
{ }% space after heading
{\thmname{#1}\thmnumber{ #2}\thmnote{: #3}}% head spec

\newtheoremstyle{problem}%
{}{}% standard spacing before and after
{}% Body style
{}{\bfseries}{}% Heading indent, font, and punctuation
{1em}% space after heading
{\fbox{\thmname{#1}\thmnumber{ #2}\thmnote{: #3}}}% head spec

\theoremstyle{box}
\newtheorem{definition}[theorem]{Definition}
\newtheorem{dfn}[theorem]{Definition}
\newtheorem*{definition*}{Definition}
\newtheorem{observation}[theorem]{Observation}
\newtheorem{remark}[theorem]{Remark}
\newtheorem{example}[theorem]{Example}
\newtheorem{question}[theorem]{Question}
\newtheorem*{prep-problems}{Preparation Problems}

%\newtheorem{problem}[theorem]{Problem}
\theoremstyle{problem}
\newtheorem{problemnum}{Problem}[chapter]
\newtheorem*{problemnum*}{Problem}
\newtheorem*{reviewnum*}{Review}
\newenvironment{problem}[1][]{\begin{problemnum}[#1]}{\end{problemnum}\nopagebreak\hrule\bigskip}
\newenvironment{problem*}[1][]{\begin{problemnum*}[#1]}{\end{problemnum*}\nopagebreak\hrule\bigskip}
\newenvironment{review*}[1][]{\begin{reviewnum*}[#1]}{\end{reviewnum*}\nopagebreak\hrule\bigskip}


% Abbreviations
\newcommand{\ii}{\ensuremath{\vec \imath}}
\newcommand{\jj}{\ensuremath{\vec \jmath}}
\newcommand{\kk}{\ensuremath{\vec k}}
\newcommand{\vv}{\ensuremath{\mathbf{v}}}
\newcommand{\colvec}[1]{\ensuremath{\begin{bmatrix}#1\end{bmatrix}}}
\DeclareMathOperator{\rank}{rank}
\DeclareMathOperator{\rref}{rref}
\DeclareMathOperator{\vspan}{span}
\DeclareMathOperator{\trace}{tr}
\DeclareMathOperator{\proj}{proj}
\DeclareMathOperator{\curl}{curl}
\newcommand{\RR}{\ensuremath{\mathbb{R}}}
% \vp is "vector prime" and corrects spacing when doing something like
% $\vec r'$ (which has the vector and prime almost touching).
% Instead, do something like $\vec r\vp$
\newcommand{\vp}{\ensuremath{^{\,\prime}}}

%Shorthand code for a link to generic wolfphram alpha
\newcommand{\wolfA}{\href{http://www.wolframalpha.com}{Wolfram Alpha}}

%The purpose of this code is to allow me to put lines in matrices so that I can create augmented matrices.
\makeatletter
\renewcommand*\env@matrix[1][*\c@MaxMatrixCols c]{%
  \hskip -\arraycolsep
  \let\@ifnextchar\new@ifnextchar
  \array{#1}}
\makeatother

\newcommand{\cl}[1]{  \begin{matrix}  #1  \end{matrix}  }
\newcommand{\bm}[1]{  \begin{bmatrix}  #1  \end{bmatrix}  }
\newcommand{\inv}{^{-1}}
\newcommand{\im}{\ensuremath{\text{im }}}
\newcommand{\R}{\mathbb{R}}
\newcommand{\blank}[1]{\raisebox{0pt}[14pt]{\rule{#1}{1pt}}}

%------------------------------------------------------------------------------------------------------------


\begin{document}
\frontmatter
\title{Multivariable Calculus}
\author{Ben Woodruff\thanks{Mathematics Faculty at Brigham Young
    University--Idaho, \url{woodruffb@byui.edu}}\\
		\valpo{Modified by Karl Schmitt\thanks{Valparaiso University -- Indiana, \url{karl.schmitt@valpo.edu}}}}
\date{Typeset on \today\\
\vfill
\includegraphics[height=1.3cm]{by-sa.eps}
\vfill
\larsonfive{With references to \emph{Calculus, Early Transcendental
    Functions}, 5th edition, by Larson and Edwards}
\thomasee{With references to \emph{Calculus}, 11th Edition by Weir, Hass, and Giordano}
\stewarts{With references to \emph{Calculus Early Transcendentals}, 7th Edition, by Stewart}}
\maketitle
\thispagestyle{empty}
\noindent\copyright{ Original 2012 Ben Woodruff.  Some Rights Reserved.\\
Modifications 2014 Karl Schmitt. Some Rights Reserved.

\bigskip

\noindent This work is licensed under the Creative Commons Attribution-Share Alike 3.0 United States License.  You may copy, distribute, display, and perform this copyrighted work, but only if you give credit to Ben Woodruff, and all derivative works based upon it must be published under the Creative Commons Attribution-Share Alike 3.0 United States License. Please attribute this work to Ben Woodruff, Mathematics Faculty at Brigham Young University--Idaho, \url{woodruffb@byui.edu}. To view a copy of this license, visit
\begin{center}
  \url{http://creativecommons.org/licenses/by-sa/3.0/us/}
\end{center}
or send a letter to Creative Commons, 171 Second Street, Suite 300, San Francisco, California, 94105, USA.}

%\chapter*{Introduction}
%This course may be like no other course in mathematics you have ever taken.  We'll discuss in class some of the key differences, and eventually this section will contain a complete description of how this course works. For now, it's just a skeleton.
%
%I received the following email about 6 months after a student took the course:
%
%\begin{quote}
%Hey Brother Woodruff,
%
%I was reading {\it Knowledge of Spiritual Things} by Elder Scott.
%I thought the following quote would be awesome to share with your
%students, especially those in Math 215 :)
%
%\begin{quote}
%Profound [spiritual] truth cannot simply be poured
%from one mind and heart to another. It takes faith
%and diligent effort. Precious truth comes a small
%piece at a time through faith, with great exertion,
%and at times wrenching struggles.
%\end{quote}
%\end{quote}
%Elder Scott's words perfectly describe how we acquire mathematical truth, as well as spiritual truth. 
%
%\section*{Teaching philosophy} Over time, I've come to view
%teaching and learning as a shared journey on which my students and I
%embark each semester. I am the subject matter expert responsible for
%providing information and guidance, setting expectations, and
%assessing how well students meet those expectations. My students are
%responsible for much hard work, including preparing in advance for
%class, participating in class activities, and doing out-of-class
%assignments, regardless of whether or not they are graded. There is
%only so much that can be conveyed in $50$ minutes, and my own personal
%experience and educational research agree that students get far less
%out of a $50$-minute lecture than their professors hope. Thus, I have
%chosen to take an approach that is more work both for you and for me
%but has been shown to produce better results. During class you
%will work on a carefully chosen series of problems designed to build
%the mathematical knowledge and experience you need to succeed. These
%problems will be done in a collaborative, small group setting where 
%you can grapple with and truly understand the material. I'll be there to support, guide, and correct
%misconceptions. Sure, I could expect you to do this alone outside of
%class, but over time I've realized a few things about working in
%groups. As a student, I usually understood something better when I
%went over it with classmates, even if I was the one who thought I
%understood it completely and explained it to a peer. As a researcher,
%I am more productive and effective when I collaborate. Friends in
%industry report that teams are increasingly used to produce the best
%results. Furthermore, having me there to help in the early stages
%ensures that we're traveling together on this journey.\\
%
%-Dr. Karl Schmitt\\
%Modified from Dr. Mitchel Keller at Washington \& Lee University
%
%\section*{Modification Notes}
%This work is based almost entirely off of Ben Woodruff's IBL textbook (see copyright information on previous pages). Some modifications have been made by Dr. Karl Schmitt at Valparaiso University to more closely match the teaching and content for Valparaiso's Calculus III course (Math 253). Planned Modifications (as of 7/22/14) include:
%\begin{itemize}
%\item Include section references for Stewart's \emph{Calculus: Early Transcendentals, 7th edition} \\
%\indent Note: Problems were relabeled (or turned off) with Thomas 11th in Chap 3 \& 4, but no Stewart entered, since it is skipped at Valpo.
%\item Include topical course objectives as defined by Valparaiso's Mathematics Dept. (these are supplemental to the previous chapter objectives)
%\item Modify introductory text to some chapters
%%\item Exclusion of the two chapters: Polar coordinates (Chapter 4) and Motion (Chapter 7). Files are included in source, but not in compiled version.
%%\item (Longterm Planned) Inclusion of additional chapter on Stokes's Theorem and Gauss's Theorem. 
%\item (Longterm Planned) Inclusion of some instructor notes/suggestions for demonstrations or examples.
%\item (Longterm Planned) Inclusion of Maple (or other software) labs/explorations. Either to supplement or replace included Sage/Wolfram
%\end{itemize}
 %
%\tableofcontents

\dominitoc \tableofcontents

\mainmatter

\chapter{Review}
\minitoc \mtcskip
\input{01-Review-215_v2}
%wrapup


\chapter{Vectors}
\minitoc \mtcskip
\input{02-Vectors_v2}
%\wrapup
%
%
\chapter{Optional Review: Conic Sections}
\minitoc \mtcskip
\input{03-Curves}
%\wrapup
%
%This breaks up the polar coordinates, and cylidrical/spherical coordinate
%Chapter from Woodruff's original version. This is because Valpo covers 
%Polar coordinates before Calc III, but needs to learn the others. So
%polar moved to a review chapter. Basically I just moved things from 
%Woodruff's original Chap 3+4 to 'New at Valpo' and 'Review at Valpo' chapters.
\chapter{Optional Review: Polar Coordinates}
\minitoc \mtcskip
\input{04-New-Coordinates_v2}
%\wrapup
%
\chapter{Parametric Eqns. and New Coordinate Systems}
\minitoc \mtcskip
\input{03-04-ALT-Parametric-Other-Coordinates_v1}

\chapter{Functions}
\minitoc \mtcskip
\input{05-Functions_v2}
%\wrapup
%
%
%
\chapter{Derivatives}
%\input{06-Derivatives}
\minitoc \mtcskip
\input{06-Derivatives-new_v2}
%\wrapup
%
\chapter{Motion}
\vspace{-1.25cm}
\minitoc \mtcskip
\input{07-Motion_v2}
%\wrapup
%
\chapter{Line Integrals}
\minitoc \mtcskip
\input{08-Line-Integrals_v2}
%\wrapup
%
%
%
\chapter{Optimization}
\minitoc \mtcskip
\input{09-Optimization_v2}
%\wrapup
%
%
\chapter{Double Integrals}
\minitoc \mtcskip
\input{10-Double-Integrals_v2}
%\wrapup
%
%%\end{document}
%
%
\chapter{Surface Integrals}
\minitoc \mtcskip
\input{11-Surface-Integrals_v2}
%\wrapup
%
\chapter{Triple Integrals}
\minitoc \mtcskip
\input{12-Triple-Integrals_v2}

%This moves the limits from an earlier chapter to their
%own chapter for coverage at the end of the semester, if 
%there's time. 
\chapter{Limits}
\minitoc \mtcskip
This extra unit returns to the idea of Limits.


\section{Limits}
In the previous chapter, we learned how to describe lots of different functions. In first-semester calculus, after reviewing functions, you learned how to compute limits of functions, and then used those ideas to develop the derivative of a function. The exact same process is used to develop calculus in high dimensions. One glitch that will prevent us from developing calculus this way in high dimensions is the epsilon-delta definition of a limit.  We'll review it briefly.  Those of you who want to pursue further mathematical study will spend much more time on this topic in future courses. 

In first-semester calculus, you learned how to compute limits of functions. Here's the formal epsilon-delta definition of a limit. 
\begin{definition}
 Let $f:\R\to\R$ be a function.
 We write $\ds \lim_{x\to c} f(x)=L$ if and only if for every $\epsilon>0$, there exists a $\delta>0$ such that $0<|x-c|<\delta$ implies $|f(x)-L|<\epsilon$.
\end{definition}
 We're looking at this formal definition here because we can compare it with the formal definition of limits in higher dimensions. The only difference is that we just put vector symbols above the input $x$ and the output $f(x)$.
\begin{definition}
 Let $\vec f:\R^n\to\R^m$ be a function.
 We write $\ds \lim_{\vec x\to \vec c} \vec f(\vec x)=\vec L$ if and only if for every $\epsilon>0$, there exists a $\delta>0$ such that $0<|\vec x-\vec c|<\delta$ implies $|\vec f(\vec x)-\vec L|<\epsilon$.
\end{definition}
We'll find that throughout this course, the key difference between first-semester calculus and multivariate calculus is that we replace the input $x$ and output $y$ of functions with the vectors $\vec x$ and $\vec y$. 
 
\begin{problem}
 For the function $f(x,y)=z$, we can write $f$ in the vector notation $\vec y=\vec f(\vec x)$ if we let $\vec x=(x,y)$ and $\vec y=(z)$. Notice that $\vec x$ is a vector of inputs, and $\vec y$ is a vector of outputs. 
 For each of the functions below, state what $\vec x$ and $\vec y$ should be so that the function can be written in the form $\vec y = \vec f (\vec x)$. \marginpar{The point to this problem is to help you learn to recognize the dimensions of the domain and codomain of the function.  If we write $\vec f:\R^n\to \R^m$, then $\vec x$ is a vector in $\R^n$ with $n$ components, and $\vec y$ is a vector in $\R^m$ with $m$ components.}  
\begin{enumerate}
 \item $f(x,y,z)=w$
 \item $\vec r(t)=(x,y,z)$
 \item $\vec r(u,v)=(x,y,z)$
 \item $\vec F(x,y)=(M,N)$
 \item $\vec F(\rho,\phi,\theta)=(x,y,z)$
\end{enumerate}
\end{problem}


You learned to work with limits in first-semester calculus without needing the formal definitions above. Many of those techniques apply in higher dimensions. 
The following problem has you review some of these technique, and apply them in higher dimensions.
\begin{problem}\marginpar{\thomasee{See 14.2: 1-30 for more practice.} \stewarts{See 14.2:5-8 for more practice.}}%
 Do these problems without using L'Hopital's rule.
%, as there is not a good substitute for L'Hopital's rule in higher dimensions. \note{check this.}
\begin{enumerate}
 \item Compute $\ds \lim_{x\to 2} x^2-3x+5$ and then $\ds\lim_{(x,y)\to (2,1)} 9-x^2-y^2$.
 \item Compute $\ds\lim_{x\to 3}\frac{x^2-9}{x-3}$ and then $\ds\lim_{(x,y)\to (4,4)} \frac{x-y}{x^2-y^2}$.
 \item Explain why $\ds\lim_{x\to 0}\frac{x}{|x|}$ does not exist. [Hint: graph the function.]
\end{enumerate}
\end{problem}



In first semester calculus, we can show that a limit does or does not exist by considering what happens from the left, and comparing it to what happens on the right.  You probably used the following theorem extensively. 
\begin{quote}
 If $y=f(x)$ is a function defined on some open interval containing $c$, then $\ds\lim_{x\to c}f(x)$ exists if and only if  $\ds\lim_{x\to c^-}f(x) = \ds\lim_{x\to c^+}f(x)$.
\end{quote}
 A limit exists precisely when the limits from every direction exists, and all directional limits are equal. In first-semester calculus, this required that you check two directions (left and right). This theorem generalizes to higher dimensions, but it becomes much more difficult to apply. 

\begin{example}
 Consider the function $\ds f(x,y)=\frac{x^2-y^2}{x^2+y^2}$.
Our goal is to determine if the function has a limit at the origin $(0,0)$. We can approach the origin along many different lines.

One line through the origin is the line $y=2x$. If we stay on this line, then we can replace each $y$ with $2x$ and then compute
$$\ds\lim_{\text{\footnotesize $\begin{array}{c}(x,y)\to(0,0)\\ y=0\end{array}$}}\frac{x^2-y^2}{x^2+y^2} 
= \lim_{x\to 0} \frac{x^2-(2x)^2}{x^2+(2x)^2}
= \lim_{x\to 0} \frac{-3x^2}{5x^2}
= \lim_{x\to 0} \frac{-3}{5}
=\frac{-3}{5}.$$
This means that if we approach the origin along the line $y=2x$, we will have a height of $-3/5$ when we arrive at the origin.
\end{example}
If the function $\ds f(x,y)=\frac{x^2-y^2}{x^2+y^2}$ has a limit at the origin, the previous example suggests that limit will be $-3/5$.
\begin{problem}
 Please read the previous example. Recall that we are looking for the limit of the function $\ds f(x,y)=\frac{x^2-y^2}{x^2+y^2}$ at the origin (0,0). 
\marginpar{You may want to look at a graph in 
\href{http://aleph.sagemath.org/?z=eJxL06jQqdS01aiIM9KtjDPS1AextEEsroKc_BLjFI00HaASXWMdY00djUoIQxMAoucONQ}{Sage}
or \href{http://wolfr.am/ioCqzX}{Wolfram Alpha} (try using the ``contour lines'' option). %http://www.wolframalpha.com/input/?i=plot+%28x%5E2-y%5E2%29%2F%28x%5E2%2By%5E2%29
 As you compute each limit, make sure you understand what that limit means in the graph.}
Our goal is to determine if the function has a limit at the origin $(0,0)$.
\begin{enumerate}
 \item In the $xy$-plane, how many lines pass through the origin $(0,0)$? Give an equation a line other than $y=2x$ that passes through the origin.  Then compute $$\ds\lim_{\text{\footnotesize $\begin{array}{c}(x,y)\to(0,0)\\ \text{your line}\end{array}$}}\frac{x^2-y^2}{x^2+y^2}
= \lim_{x\to 0} \frac{x^2-(?)^2}{x^2+(?)^2}=\ldots.$$
 \item Give another equation a line that passes through the origin.  Then compute $$\ds\lim_{\text{\footnotesize $\begin{array}{c}(x,y)\to(0,0)\\ \text{your line}\end{array}$}}\frac{x^2-y^2}{x^2+y^2}.$$
 \item Does this function have a limit at $(0,0)$? Explain. \marginpar{\thomasee{See 14.2: 41-50 for more practice.}\larsonfive{See Larson 13.2:23--36 and example 4 for more practice.} \stewarts{See 14.2: 9-12}}%
\end{enumerate}
\end{problem}


The theorem from first-semester calculus generalizes as follows.
\begin{quote}
 If $\vec y=\vec f(\vec x)$ is a function defined on some open region containing $\vec c$, then $\ds\lim_{\vec x\to \vec c}\vec f(\vec x)$ exists if and only if the limit exists along every possible approach to $\vec c$ and all these limits are equal.
\end{quote}
There's a fundamental problem with using this theorem to check if a limit exists. Once the domain is 2-dimensional or higher, there are infinitely many ways to approach a point. There is no longer just a left and right side. To prove a limit exists, you must check infinitely many cases --- that takes a really long time.  The real power to this theorem is it allows to show that a limit does not exist.  All we have to do is find two approaches with different limits.



\begin{problem}
\marginpar{See \href{http://aleph.sagemath.org/?z=eJyVVF1r2zAUfc-vuKQFy7PS2QndoCBY2dtgMFjfShtubLnW6lhCUlqrv37XlvOxtdtYEoKlc3TOufcKn50BfNFNBzcWn5Tj8FU5N_yMUfBZt618kBy-G6u6B1jmRTGbPaFlSc9Dks4-qc5Li6WfVbKGNauF6szOrze6Z7SDu9YL1r8L6XvW3y-zcL9MUz6D8WP8W-Sc5ymHFjeyFfNvmvSv4JyRWyrO5xyeVeUbUeQHFTTGaiyb4g2xRX9Qup5oUJBc-LvU8g0pSv9aa_laa5JKr8aHPuchF8aPi6GFHhq_bVlyDW5naywl7eoNbtoADTpAaNVWeUAPviFsKB9UPSysBOWg0wTS2aqSHZQNdg-0TU-61TayLpJ0MIv2diDQAHifLwr6y4oYMExAoHgEhAOwTyVIsvN6Z9em1Z7VPErx6SQvt2hE8kP6hI_eG7Tixu4IiMecuJx6MRa0jpWIWBFjY1_SeFQkVlYJd-pFilXOX7StpBWreLqmsnCokB3mzI9zmrp8EjwTamtaVQ6WQ_CwwH30ffLouWmxfEymohv9zE5yZpNYRLGXTqC1xGFR6ra44-M13a-XcZ1mEy2fzEYi3ebjxsA85eV8UURCzov0aJgJL3u_qti8p9t19Mkuislqj4f5r_oj4wR_me_lCZkcjBiaQ2j9W3NAG6TeBZFffOADJzbEiY-XFJpy_d9QTCYMWtxKb1UZB0J3EXnNgkB6EcDhVm1VJ449ozX24tgy36jysZPOidUf52f-qYLOyNKvLXqlxW3B6XuXzn4C8NyNiQ}{Sage}.}%
\marginpar{\thomasee{See 14.2: 41-50 for more practice.}\larsonfive{See Larson 13.2:9--36 for more practice.}\stewarts{See 14.2: 13-22 for a mix of functions with limits and without}}%
 Consider the function $\ds f(x,y) = \frac{xy}{x^2+y^2}$.  Does this function have a limit at $(0,0)$?  Examine the function at $(0,0)$ by considering the limit as you approach the origin along several lines. 
\end{problem}

In all the examples above, we considered approaching a point by traveling along a line. Even if a function has a consistent limit along EVERY line, that is not enough to guarantee the function has a limit. The theorem requires EVERY approach, which includes parabolic approaches, spiraling approaches, and more. For our purposes, checking along straight lines will do.  If you are interested in seeing an example of a function $f(x,y)$ so that the limit at $(0,0)$ along every straight line $y=mx$ exists and equals 0, but the function has no limit at $(0,0)$, then please ask.  Alternately, if this interests you, try coming up with an example yourself, and then come show me when you get it. This is a fun challenge.

\begin{problem*}[Challenge]
 Give an example of a function $f(x,y)$ so that the limit at $(0,0)$ along every straight line $y=mx$ exists and equals 0.  However, show that the function has no limit at $(0,0)$ by considering an approach that is not a straight line.
\end{problem*}
%\wrapup

\begin{comment}
\end{comment}
\end{document}
