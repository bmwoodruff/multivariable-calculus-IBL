%Template file for LaTeX -- Lesson Plans with Instructor and Student Versions

\documentclass{article}
\usepackage{setspace}
\usepackage{amssymb}
\usepackage{amsmath}
\usepackage{amsthm}
\usepackage{hyperref}
\usepackage{enumitem}
\usepackage{fancyhdr}
\usepackage{lastpage}
% Set up page size
% All margin dimensions measured from a point one inch from top and side
% of page.  Dimensions shrink by about 2 percent

%Comment out 2nd line for instructor comments to be included.
\newcommand{\instructor}[1]{#1}
%\newcommand{\instructor}[1]{}

\pagestyle{fancy}
\lhead{MATH 253}
\rhead{Schmitt - Spring 15}
%\cfoot{Page \thepage}
\lfoot{}

\fancyhfoffset[L]{0in}
\fancyhfoffset[R]{0in}

\usepackage[left=1in,right=1in,top=1in,bottom=1in]{geometry}

\theoremstyle{plain}
\newtheorem*{theorem}{Theorem}
\newtheorem*{corollary}{Corollary}
\newtheorem*{lemma}{Lemma}
\newtheorem*{proposition}{Proposition}

\theoremstyle{definition}
\newtheorem*{definition}{Definition}
\newtheorem*{example}{Example}
\newtheorem*{note}{Note}
\newtheorem*{aside}{Aside}

\theoremstyle{remark}
\newtheorem*{notation}{Notation}

%\renewcommand{\labelenumii\arabic{enumi}}
%\renewcommand{\labelenumii}{\labelenumi(\roman{enumii}) }

%Modify the Chapter here
\chead{\Large{Review Unit 10}}
\cfoot{Page \thepage \ of \ \pageref{LastPage}}

\begin{document}
\vskip-2pt
\noindent \large Learning Objective:
\normalsize
\vskip0.15in
$\bullet$ perform double integrals in standard coordinate systems\\
\indent \emph{Note: Page 2 and 3 has CHANGE of coordinate systems/variables}
\vskip0.15in
\hrule
\vspace{0.1in}
\large \noindent Key Equations:
\normalsize

\vspace{1in}
\hrule
\vspace{0.1in}
\large \noindent Notes (3+ sentences or ideas):
\normalsize
\vspace{1.75in}
\hrule
\vspace{0.1in}

\large \noindent Examples with explanations (2+):\\
\emph{Include at least one example of each}

\newpage
\noindent \large Learning Objective:
\normalsize
\vskip0.15in
$\bullet$ perform double integrals by changing coordinate systems
\emph{Note: page 3 has changing order (x vs. y) of integration.}
\vskip0.15in
\hrule
\vspace{0.1in}
\large \noindent Key Equations\\
\normalsize

\vspace{1in}
\hrule
\vspace{0.1in}
\large \noindent Notes (3+ sentences or ideas):
\normalsize
\vspace{1.75in}
\hrule
\vspace{0.1in}

\large \noindent Examples with explanations (2+):

\newpage
\noindent \large Learning Objective:
\normalsize
\vskip0.15in
$\bullet$ Understand Fubini's Theorem and change the order of integration when appropriate
\vskip0.15in
\hrule
\vspace{0.1in}
\large \noindent Key Equations:
\normalsize

\vspace{1in}
\hrule
\vspace{0.1in}
\large \noindent Notes (3+ sentences or ideas):
\normalsize
\vspace{1.75in}
\hrule
\vspace{0.1in}

\large \noindent Examples with explanations (2+):\\
\normalsize

\newpage
\noindent \large Learning Objective:
\normalsize
\vskip0.15in
$\bullet$ apply Green's Theorem to evaluate line integrals over simple closed curves in the plane
\vskip0.15in
\hrule
\vspace{0.1in}
\large \noindent Key Equations:
\normalsize

\vspace{1in}
\hrule
\vspace{0.1in}
\large \noindent Notes (3+ sentences or ideas):
\normalsize
\vspace{1.75in}
\hrule
\vspace{0.1in}

\large \noindent Examples with explanations (2+):\\
\normalsize


\newpage
%\noindent \large Learning Objective:
%\normalsize
%\vskip0.1in
%\noindent Other Important Concept(s) or Idea(s) (at least one)\emph{--From this Chapter}:\\
%\vspace{0.75in}
%%What students will be doing in-class for the day/week
%\hrule
%\vspace{0.1in}
%\large \noindent Key Equations:
%\normalsize
%
%\vspace{1.25in}
%\hrule
%\vspace{0.1in}
%\large \noindent Notes (3+ sentences or ideas):
%\normalsize
%\vspace{1.75in}
%\hrule
%\vspace{0.1in}
%
%\large \noindent Examples with explanations (2+):

\large \noindent Two ADDITIONAL examples, expanding on any of the three main objectives, with explanations (2+):\\
You will not get credit for this page if you do not include explanations.

\end{document}
