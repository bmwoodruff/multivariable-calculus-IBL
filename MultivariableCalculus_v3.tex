
\documentclass[letterpaper,oneside]{book}%

%Several If statements. These control various comments:
%  True/False -> if true is uncommented, these are INCLUDED.
% instructor -> notes for instructor
% notes -> Footnotes
% thomas -> Problems/References to Thomas's Calculus 11th edition
% larson -> Problem/References to Larson's Calculus, 5th edition
% stewart -> Problems/References to Stewart's Calculus: Early Transcendentals, 7th edition
% bmw -> Comments & Hints from Ben Woodruff
% valpo -> Comments specifically for Valparaiso University Students
% intro -> Includes two pages of 'framing' for inquiry-based learning from Woodruff and Schmitt

\newif\ifinstructor
%\instructortrue
\instructorfalse

\newif\ifnotes
%\notestrue
\notesfalse

\newif\ifthomas
%\thomastrue
\thomasfalse

\newif\iflarson
%\larsontrue
\larsonfalse

\newif\ifstewart
\stewarttrue
%\stewartfalse

\newif\ifbmw
%\bmwtrue
\bmwfalse

\newif\ifvalpo
\valpotrue
%\valpofalse

\newif\ifIntro
\Introtrue
%\Introfalse


\usepackage[left=1in,right=2.75in,top=1in,bottom=1in]{geometry}
\marginparwidth 1.75in

%\let\oldmarginpar\marginpar
%\renewcommand\marginpar[1]{\-\oldmarginpar{\raggedright\footnotesize #1}}
%\renewcommand\marginpar[1]{\-\oldmarginpar[\raggedleft\footnotesize #1]{\raggedright\footnotesize #1}}

%There's probably a whole slew of this stuff that can/should be cut to clean up the TeX code.
% 7/31/16 -- Karl Schmitt

\usepackage{tabls}
\usepackage{booktabs}
\usepackage{amsmath}
\usepackage{amssymb}
\usepackage{amsthm}
\usepackage{amsfonts}
\usepackage{multicol}
\usepackage[shortlabels]{enumitem}
\usepackage{microtype}
\usepackage{multirow}
\usepackage{comment}
\usepackage{graphicx}
\usepackage{wrapfig}
\usepackage[utf8]{inputenc}
\usepackage[upright]{fourier}
\usepackage{tikz}
\usetikzlibrary{positioning}
\usetikzlibrary{matrix,arrows,decorations.pathmorphing}


\usepackage{color}
\definecolor{darkblue}{rgb}{0, 0, .6}

\usepackage{hyperref}
\hypersetup{
	colorlinks=true,
	linkcolor=darkblue,
	anchorcolor=darkblue,
	citecolor=darkblue,
	urlcolor=darkblue,
}

\usepackage{minitoc}

%================================================================================================
% Define some chapter related macros.
% Wrap-Ups -- called at the end. Currently inside if-stuff to allow either to be picked.
% Namings
%-----------------------------------------------------------

\renewcommand{\chaptername}{Unit} %Because I like 'unit' better than chapter. --KS
%When possible I've used \chaptername when the term unit or chapter should occur, so this can be removed and it'll
% go back to reading as 'chapters'. -- KS
\newcommand{\chpname}{unit} % Since I'm not exactly sure what '\chaptername' will actually return. Will use this, and test when I compile.
\setcounter{chapter}{-1} %So the Review chapter is chapter 0.


%The next two commands are retain for compatibility issues. I'm planning to remove this explicit split in the Fall-2016 version.
\newcounter{unitday}[chapter]
\newcommand{\uday}{\newpage \LARGE \center  Day \theunitday \normalsize \flushleft \stepcounter{unitday} \vskip0.1in}

\newcommand{\wrapup}{
\bmw{\section{Wrap Up}
Once you have finished the problems in this \chaptername and feel comfortable with the ideas, create a short one page lesson plan that contains examples of the key ideas.  You will get a chance to teach from this lesson plan prior to taking the exam. Then log on to Brainhoney and download the quiz. Once you have taken the quiz, you can upload your work back to brainhoney and then download the key to see how you did. If you still need to work on mastering some of the ideas, please do so and then demonstrate your mastery though the quiz corrections.}

%I liked Ben's wrapup for his Diff-Eq book better. So I just stole it and modified it a bit!
\valposhort{\section{Wrap Up}
This concludes the \chaptername .  Look at the objectives at the beginning of the \chpname . Can you now do all the things you were promised? \\
\instructor{TeX files for these are included in a separate folder in the GitHub Project}
\note{Those files should be cleaned up an generalized, i.e. remove specific instructor/course number from them. Possibly also make a master tex file to compile them all}

\textbf{Review Guide Creation}\\

Your assignment: organize what you've learned into a small collection of examples that illustrates the key concepts. I'll call this your \chpname\ review guide. I'll provide you with a template which includes the \chpname's key concepts from the objectives at the beginning. Once you finish your review guide, scan it into a PDF document (use any scanner on campus or photo software) and upload it to Gradescope. \instructor{Gradescope is a cool grading platform! \url{www.gradescope.com} } \\

As you create this review guide, consider the following:
\begin{itemize}
% \item On the class period after making this plan, you'll have 20 minutes in class where you will get to teach a peer your examples. If you keep the examples simple, you'll be able to fully review the entire chapter.
 \item Before each Celebration of Knowledge \instructor{This is just an exam} we will devote a class period to review. With well created lesson plans, you will have 4-8 pages(for 2-4 Chapters) to review for each, instead of 50-100 problems.
 \item Think ahead 2-5 years. If you make these lesson plans correctly, you'll be able to look back at your lesson plans for this semester. In about 20-25 pages, you can have the entire course summarized and easy for you to recall.
\end{itemize}
} %endvalpo
} %end wrapup

%======================================================================



%*****************************************************************************************************************************************

%=========================================================================
% This section defines a large set of commands within IF-ELSE statements
% What is turned on/off is set in the first section of this TeX code.
% Note that each includes 'null' commands in the ELSE part to allow the text to compile even when something is unwanted.
%=========================================================================

%Note Both of the instructor and 'notes' appear in colors (blue, red respectively) to make them stand out more in the digital copy.

% Instructor-specific material (answers, helps, etc.)
\ifinstructor
  \newcommand{\instructor}[1]{\marginpar{\textcolor{blue}{\textbf{Instructor: }#1}}}
\else
  \newcommand{\instructor}[1]{}
\fi

%These are notes between authors of the text, primarily Ben Woodruff and Jason Grout
\ifnotes
\renewcommand{\thefootnote}{\roman{footnote}}
\newcommand{\note}[1]{\footnote{#1}\marginpar{\fbox{\textbf{\thefootnote}}}}
\else
\newcommand{\note}[1]{}
\fi

%Specific comments, problems, etc relevant to Ben Woodruff's Class
%\ifbmw
%	\newcommand{\bmw}[1]{#1}
%	\newcommand{\marginparbmw}[1]{\bmw{\marginpar{#1}}}
%\else
%	\newcommand{\bmw}[1]{}
%	\newcommand{\marginparbmw}[1]{}
%\fi
%	
%%Notes for Students @ Valparaiso University
%%Two versions of each command are included to allow notes without the 'pre-label'
%\ifvalpo
%	\newcommand{\valpo}[1]{\textbf{To Valpo Students:} #1}
%	\newcommand{\valposhort}[1]{#1}
%\else
%	\newcommand{\valpo}[1]{}
%	\newcommand{\valposhort}[1]{}
%\fi	

%----------------------------------------------------------------------------------------------------------
%The next three (or more) are adding refences to specific, copyrighted textbooks for additional problems.
%Textbook editions are included to allow multiple versions of a text to be used.
%Two versions of each command are included to allow notes without the 'pre-label'
%The 'short' version has the longer command because it is used less frequently.

% Notes for Thomas, 11th edition
%\ifthomas
%	\newcommand{\thomasee}[1]{Thomas: #1}
%	\newcommand{\thomaseeshort}[1]{#1}
%\else
%	\newcommand{\thomasee}[1]{}
%	\newcommand{\thomaseeshort}[1]{}
%\fi
%
%%Notes for Larson, 5th Edition
%\iflarson
%	\newcommand{\larsonfive}[1]{Larson: #1}
%	\newcommand{\larsonfiveshort}[1]{#1}
%\else
%	\newcommand{\larsonfive}[1]{}
%	\newcommand{\larsonfiveshort}[1]{}
%\fi
%
%Notes for Stewart, 7th Edition
%\ifstewart
%	\newcommand{\stewarts}[1]{Stewart: #1}
%	\newcommand{\stewartsshort}[1]{#1}
%\else
%	\newcommand{\stewarts}[1]{}
%	\newcommand{\stewartsshort}[1]{}
%\fi

%******************************************************************************************************************


%--------------------------------------------------------------------------------------------------
% Setting up the Theorem/Problem Styles and macros.
%--------------------------------------------------------------------------------------------------

%-------------------
% Setting the Styles
%-------------------
\newtheoremstyle{box}%
{}{}% standard spacing before and after
{}% Body style
{}{\bfseries}{.}% Heading indent, font, and punctuation
{ }% space after heading
{\thmname{#1}\thmnumber{ #2}\thmnote{: #3}}% head spec

\newtheoremstyle{problem}%
{}{}% standard spacing before and after
{}% Body style
{}{\bfseries}{}% Heading indent, font, and punctuation
{\newline}% space after heading
{\fbox{\thmname{#1}\thmnumber{ #2}\thmnote{: #3}}}% head spec

%========================================================================================
% Three types of 'theorem' styles are used for various problems/theorems/definitions.
%========================================================================================
%
% Note: Some of these may not actually be used and should be removed.
%

\theoremstyle{plain}
\newtheorem{theorem}{Theorem}[chapter]
\newtheorem*{theorem*}{Theorem}
\newtheorem{lemma}[theorem]{Lemma}
\newtheorem*{lemma*}{Lemma}
\newtheorem{proposition}[theorem]{Proposition}
\newtheorem{corollary}[theorem]{Corollary}


%Attempting to Comment out some of this, and see what can actually be trimmed from the code.
% Done on 7/31/16
\theoremstyle{box}
\newtheorem{definition}[theorem]{Definition}
%\newtheorem{dfn}[theorem]{Definition}[chapter]
\newtheorem*{definition*}{Definition}
\newtheorem{observation}[theorem]{Observation}
\newtheorem{remark}[theorem]{Remark}
\newtheorem{example}[theorem]{Example}

%\newtheorem{question}[theorem]{Question}
%\newtheorem*{prep-problems}{Preparation Problems}
%\newtheorem{problem}[theorem]{Problem}

%=====
% On a suggestion from a recent conference on pedagogy, it was suggested that we really shouldn't call what we do ``problems'', but rather
% 'exercises' since we are practicing what we are learning to get better. So... even though all the actual pages have \begin{problem}
% I've adjusted the macro here to make them read as 'exercise' 
% In a similar vein, rather than have 'optional' exercises, I have 'challenge' exercises. 
% Karl
%=====

\theoremstyle{problem}
\newtheorem{problemnum}{Exercise}[chapter]
\newtheorem{challengenum}{Challenge Exercise}[chapter]
\newtheorem*{problemnum*}{Exercise}
\newtheorem*{reviewnum*}{Review}
\newenvironment{problem}[1][]{\begin{problemnum}[#1]}{\end{problemnum}\nopagebreak\hrule\bigskip}
\newenvironment{problem*}[1][]{\begin{problemnum*}[#1]}{\end{problemnum*}\nopagebreak\hrule\bigskip}
\newenvironment{review*}[1][]{\begin{reviewnum*}[#1]}{\end{reviewnum*}\nopagebreak\hrule\bigskip}
\newenvironment{challenge}[1][]{\smallskip\hrule\begin{challengenum}[#1]}{\end{challengenum}\nopagebreak\hrule\bigskip}

%--------------------------------------

%The purpose of this code is to allow me to put lines in matrices so that I can create augmented matrices.
\makeatletter
\renewcommand*\env@matrix[1][*\c@MaxMatrixCols c]{%
  \hskip -\arraycolsep
  \let\@ifnextchar\new@ifnextchar
  \array{#1}}
\makeatother

%---------------------------------------


%-----------------------------------------------------------------------------------------
% Abbreviations
%---------------------------------------
%
% Note: There seems to be a lot of duplication in this.
% References/abbreviations in specific chapter files should probably be cleaned up/standardized.
%

\newcommand{\myscale}{1}
\newcommand{\ds}{\displaystyle}
\newcommand{\dfdx}[1]{\frac{d#1}{dx}}
\newcommand{\ddx}{\frac{d}{dx}}

\newcommand{\ii}{\ensuremath{\vec \imath}}
\newcommand{\jj}{\ensuremath{\vec \jmath}}
\newcommand{\kk}{\ensuremath{\vec k}}
\newcommand{\vv}{\ensuremath{\mathbf{v}}}
\newcommand{\RR}{\ensuremath{\mathbb{R}}}
\newcommand{\R}{ \ensuremath{\mathbb{R}}}
\newcommand{\inv}{^{-1}}
\newcommand{\im}{\ensuremath{\text{im }}}

\newcommand{\colvec}[1]{\ensuremath{\begin{bmatrix}#1\end{bmatrix}		}}
%\newcommand{\cl}[1]{ \ensuremath{ \begin{matrix}  #1  \end{matrix}  	}}
%\newcommand{\bm}[1]{ \ensuremath{ \begin{bmatrix}  #1  \end{bmatrix}  }}

\DeclareMathOperator{\rank}{rank}
\DeclareMathOperator{\rref}{rref}
\DeclareMathOperator{\vspan}{span}
\DeclareMathOperator{\trace}{tr}
\DeclareMathOperator{\proj}{proj}
\DeclareMathOperator{\curl}{curl}

\newcommand{\blank}[1]{\raisebox{0pt}[14pt]{\rule{#1}{1pt}}}

% \vp is "vector prime" and corrects spacing when doing something like
% $\vec r'$ (which has the vector and prime almost touching).
% Instead, do something like $\vec r\vp$
\newcommand{\vp}{\ensuremath{^{\,\prime}}}

%Shorthand code for a link to generic wolfphram alpha
\newcommand{\wolfA}{\href{http://www.wolframalpha.com}{Wolfram Alpha} }

%------------------------------------------------------------------------------------------------------------


\begin{document}

\frontmatter
\title{Multivariable Calculus}
\author{Ben Woodruff
				\thanks{Mathematics Faculty at Brigham Young University--Idaho, \url{woodruffb@byui.edu}}\\
				Modified by Karl R. B. Schmitt
				\thanks{Valparaiso University -- Indiana, \url{karl.schmitt@valpo.edu} }
				}
\date{Typeset on \today\\
	\vfill
	\includegraphics[height=1.3cm]{by-sa.eps}
	\vfill
	\larsonfiveshort{With references to \emph{Calculus, Early Transcendental Functions}, 5th edition, by Larson and Edwards\\}
	\thomaseeshort{With references to \emph{Calculus}, 11th Edition by Weir, Hass, and Giordano\\}
	\stewartsshort{With references to \emph{Calculus Early Transcendentals}, 7th Edition, by Stewart\\}
	}

\maketitle

\thispagestyle{empty}

\noindent\copyright{ Original 2012- Ben Woodruff.  Some Rights Reserved. Modifications 2014- Karl Schmitt. Some Rights Reserved.\\

\bigskip

\noindent This work is licensed under the Creative Commons Attribution-Share Alike 3.0 United States License.  You may copy, distribute, display, and perform this copyrighted work, but only if you give credit to Ben Woodruff, and all derivative works based upon it must be published under the Creative Commons Attribution-Share Alike 3.0 United States License. Please attribute this work to Ben Woodruff, Mathematics Faculty at Brigham Young University--Idaho, \url{woodruffb@byui.edu}. 

\noindent Furthermore, this deriviative with edits and modifications by Karl R. B. Schmitt similarly requires that you give additional editing/modification credit to Karl R. B. Schmitt. Please attribute these modifications and edits to Karl R. B. Schmitt, Mathematics and Statistics Faculty at Valparaiso University--Indiana, \url{karl.schmitt@valpo.edu}

To view a copy of this license, visit
\begin{center}
  \url{http://creativecommons.org/licenses/by-sa/3.0/us/}
\end{center}
or send a letter to Creative Commons, 171 Second Street, Suite 300, San Francisco, California, 94105, USA.}

%======================================================
% Some optional introductory comments from Ben Woodruff and Karl Schmitt about what taking (and inquiry-based) course may be like.
% Left 'on' as a default.
%======================================================

\ifIntro
\chapter*{Introduction}
\section*{Mathematical Truths}
From Dr. Woodruff:\\

\noindent This course may be like no other course in mathematics you have ever taken.  We'll discuss in class some of the key differences, and eventually this section will contain a complete description of how this course works. For now, it's just a skeleton.\\

\noindent I received the following email about 6 months after a student took the course:

\begin{quote}
Hey Brother Woodruff,

I was reading {\it Knowledge of Spiritual Things} by Elder Scott.
I thought the following quote would be awesome to share with your
students, especially those in Math 215 :)

\begin{quote}
Profound [spiritual] truth cannot simply be poured
from one mind and heart to another. It takes faith
and diligent effort. Precious truth comes a small
piece at a time through faith, with great exertion,
and at times wrenching struggles.
\end{quote}
\end{quote}
Elder Scott's words perfectly describe how we acquire mathematical truth, as well as spiritual truth. 

\newpage

\section*{Teaching philosophy} 
From Dr. Schmitt:\\

\noindent Over time, I've come to view teaching and learning as a shared journey on which my students and I
embark each semester. I am the subject matter expert responsible for
providing information and guidance, setting expectations, and
assessing how well students meet those expectations. My students are
responsible for much hard work, including preparing in advance for
class, participating in class activities, and doing out-of-class
assignments, regardless of whether or not they are graded. There is
only so much that can be conveyed in $50$ minutes, and my own personal
experience and educational research agree that students get far less
out of a $50$-minute lecture than their professors hope. Thus, I have
chosen to take an approach that is more work both for you and for me
but has been shown to produce better results. During class you
will work on a carefully chosen series of problems designed to build
the mathematical knowledge and experience you need to succeed. These
problems will be done in a collaborative, small group setting where 
you can grapple with and truly understand the material. I'll be there to support, guide, and correct
misconceptions. Sure, I could expect you to do this alone outside of
class, but over time I've realized a few things about working in
groups. As a student, I usually understood something better when I
went over it with classmates, even if I was the one who thought I
understood it completely and explained it to a peer. As a researcher,
I am more productive and effective when I collaborate. Friends in
industry report that teams are increasingly used to produce the best
results. Furthermore, having me there to help in the early stages
ensures that we're traveling together on this journey.\\

-Dr. Karl Schmitt\\
Modified with permission from Dr. Mitchel Keller at Washington \& Lee University

\fi

%--------------------------------------------------------------------------------------------------------------------------
% Some hidden modification notes. Last updated Fall 2014
%---------------

\section*{Modification Notes}
This work is based almost entirely off of Ben Woodruff's IBL textbook (see copyright information on previous pages). Some modifications have been made by Dr. Karl Schmitt at Valparaiso University to more closely match the teaching and content for Valparaiso's Calculus III course (Math 253). In particular several chapters/units have been split apart or reorganized.
%\begin{itemize}
%\item Include section references for Stewart's \emph{Calculus: Early Transcendentals, 7th edition} \\
%\indent Note: Problems were relabeled (or turned off) with Thomas 11th in Chap 3 \& 4, but no Stewart entered, since it is skipped at Valpo.
%\item Include topical course objectives as defined by Valparaiso's Mathematics Dept. (these are supplemental to the previous chapter objectives)
%\item Modify introductory text to some chapters
%%\item Exclusion of the two chapters: Polar coordinates (Chapter 4) and Motion (Chapter 7). Files are included in source, but not in compiled version.
%%\item (Longterm Planned) Inclusion of additional chapter on Stokes's Theorem and Gauss's Theorem. 
%\item (Longterm Planned) Inclusion of some instructor notes/suggestions for demonstrations or examples.
%\item (Longterm Planned) Inclusion of Maple (or other software) labs/explorations. Either to supplement or replace included Sage/Wolfram
%\end{itemize}
 %

%----------------------------------------------------------------------------------------------------------------------

\dominitoc \tableofcontents

\mainmatter

\chapter{Review: Calculus I \& II}
\minitoc \mtcskip
\input{01-Review-v3}
\clearpage
%\wrapup


\chapter{Vectors}
\minitoc \mtcskip
\noindent 
In this unit you will learn how to...
\begin{enumerate}
\item Define, draw, and explain what a vector is in 2 and 3 dimensions.
\item Add, subtract, multiply (scalar, dot product, cross product) vectors. Be able to illustrate each operation geometrically.
\item Use vector products to find angles, length, area, projections, and work.
\item Use vectors to give equations of lines and planes, and be able to draw lines and planes in 3D.
\end{enumerate}

\clearpage

\normalsize
\section{Vectors and Lines}
\textbf{Topical Objectives:}
\begin{itemize}
\item understand vectors as quantities having length and direction, independent of position
\item express curves with parametric (vector) equations
\end{itemize}

Learning to work with vectors will be key tool we need for our work in high dimensions.  Let's start with some problems related to finding distance in 3D, drawing in 3D, and then we'll be ready to work with vectors.
`

\begin{problem}
\instructor{\\Pre-Class Recommended\\}
To find the distance between two points $(x_1,y_1)$ and $(x_2,y_2)$ in the plane, we create a triangle connecting the two points.  The base of the triangle has length $\Delta x=(x_2-x_1)$ and the vertical side has length $\Delta y=(y_2-y_1)$. The Pythagorean theorem gives us the distance between the two points as $\sqrt{\Delta x^2+\Delta y^2}=\sqrt{(x_2-x_1)^2+(y_2-y_1)^2}$.\\

\begin{enumerate}
\item Draw a 3-D axis then plot the points $(1,5)$ and $(2,3)$
\item Sketch the triangle that connects these two points and the origin -- $(0,0)$
\item Find $\Delta x$ and $\Delta y$
\item Now extend your picture into 3 and use it to show that the distance between two points $(x_1,y_1,z_1)$ and $(x_2,y_2,z_2)$ in 3-dimensions is $\sqrt{\Delta x^2+\Delta y^2+\Delta z^2}=\sqrt{(x_2-x_1)^2+(y_2-y_1)^2+(z_2-z_1)^2}$.
\item Use this formula to find the distance between the points $(1,5,2)$ and $(2,3,3)$
\end{enumerate}
\end{problem}


\begin{problem}
\instructor{\\Pre-Class Recommended\\}
\marginpar{
	\thomasee{See 12.1:41-58.} 
	\stewarts{See 12.1:11-18}
	}%end margin
Recall that a circle (or sphere in 3-D) is just a collection of points which are all equal distance away from a center point.
\begin{enumerate}
\item If the center of a circle is the point $(1,5)$ what is the equation for a circle which passes through the point $(2,3)$?
\item Find the distance between the two points $P=(2,3,-4)$ and $Q=(0,-1,1)$. 
\item Now find an equation of the sphere passing though point $Q$ whose center is at $P$. \\
Hint: Each point on the surface of a sphere is $r$ distance away from the center.
\end{enumerate}
\end{problem}


%\begin{problem}
%\marginpar{
	%\thomasee{See 12.1:1-40.} 
	%\stewarts{12.1:23-34}
	%}
%For each of the following, construct a rough sketch of the set of points in space (3D) satisfying:
%\begin{enumerate}
%\item $2\leq z\leq 5$
%\item $x=2,y=3$
%\item $x^2+y^2+z^2=25$
%\end{enumerate}
%\end{problem}

\begin{definition}
A vector is a magnitude in a certain direction.  
If $P$ and $Q$ are points, then the vector $\vec{PQ}$ is the directed line segment from $P$ to $Q$. This definition holds in 1D, 2D, 3D, and beyond.  
If $V=(v_1,v_2,v_3)$ is a point in space, then to talk about the vector $\vec v$ from the origin $O$ to $V$ we'll use any of the following notations:
$$\vec v = \vec{OV}=\left<v_1,v_2,v_3\right> 
= v_1\mathbf{i}+v_2\mathbf{j}+v_3\mathbf{k} 
= (v_1,v_2,v_3) 
= \begin{pmatrix}v_1\\ v_2\\ v_3\end{pmatrix}
.$$
The entries of the vector are called the $x$, $y$, and $z$ (or i, j, k) components of the vector. 
\end{definition}
\instructor{There are no problems actually exploring the idea of vectors NOT from the origin, i.e. between several points. This should be added or tweaked into the existing problems}
Note that $(v_1,v_2,v_3)$ could refer to either the point $V$ or the vector $\vec v$. The context of the problem we are working on will help us know if we are dealing with a point or a vector.

\begin{definition}
Let $\mathbb{R}$ represent the set real numbers. Real numbers are actually 1D vectors.\\
Let $\mathbb{R}^2$ represent the set of vectors $(x_1,x_2)$ in the plane.\\
Let $\mathbb{R}^3$ represent the set of vectors $(x_1,x_2,x_3)$ in space. There's no reason to stop at 3, so let $\mathbb{R}^n$ represent the set of vectors $(x_1,x_2,\ldots,x_n)$ in $n$ dimensions.
\end{definition}
In first semester calculus and before, most of our work dealt with problem in $\mathbb{R}$ and $\mathbb{R}^2$. Most of our work now will involve problems in $\mathbb{R}^2$ and $\mathbb{R}^3$. We've got to learn to visualize in $\mathbb{R}^3$.

\begin{definition}\label{def:vecadd}
Suppose $\vec x=\left<x_1,x_2,x_3\right>$ and $\vec y=\left<y_1,y_2,y_3\right>$ are two vectors in 3D, and $c$ is a real number. We define vector addition and scalar multiplication as follows:
\begin{itemize}
\item Vector addition: $\vec x+\vec y = (x_1+y_1,x_2+y_2,x_3+y_3)$ (add component-wise).
\item Scalar multiplication: $c\vec x = (cx_1,cx_2,cx_3)$.
\end{itemize}
\end{definition}

\begin{problem}
\marginpar{
	\thomasee{See 12.2:23-24.} 
	\stewarts{See 12.2:5-6}
	}
Consider the vectors $\vec u=\langle 1,2 \rangle$ and $\vec v=\left<3,1\right>$.  Draw $\vec u$, $\vec v$, $\vec u+\vec v$, and $\vec u-\vec v$ with their tail placed at the origin.  Then draw $\vec v$ with its tail at the head of $\vec u$. 
\end{problem}

\begin{problem}\label{prob:donkey}
\marginpar{
	\thomasee{See 11.1: 3,4.}
	}
Consider the vector $\vec v=\langle 3,-1 \rangle $.\\
\begin{enumerate}
\item Draw $\vec v$, $-\vec v$, and $3\vec v$.
\end{enumerate}
\noindent Suppose a donkey travels along the path given by $(x,y)=\vec v t = (3t,-t)$, where $t$ represents time.
\begin{enumerate}[resume]
\item At the following times, where is the donkey?
	\begin{enumerate}
		\item $t=0$ ?
		\item $t=1$ ?
		\item $t=2$ ?
	\end{enumerate}
\item Draw an axis, then sketch the path followed by the donkey. 
\item Where is the donkey at time $t=0,1,2$? Put markers with labels on your graph to show the donkey's location.
\end{enumerate}
\end{problem}

\vskip0.2in

In a previous problem (\ref{prob:donkey}) you encountered $(x,y)=(3t,-t)$.  This is an example of a function where the input is $t$ and the output is a vector $(x,y)$.  For each input $t$, you get a single vector output $(x,y)$. Such a function is called a parametrization of the donkey's path. Because the output is a vector, we call the function a vector-valued function. Often, we'll use the variable $\vec r$ to represent the radial vector $(x,y)$, or $(x,y,z)$ in 3D.  So we could rewrite the position of the donkey as $\vec r(t)=(3,-1)t$. We use $\vec r$ instead of $r$ to remind us that the output is a vector.

Let's look at another, similar problem.

\begin{problem}\label{prob:horseline}
\marginpar{
	\thomasee{See 12.2: 1.}
	}
Suppose a horse races down a path given by the vector-valued function $\vec r(t) = (1,2)t+(3,4)$. (Remember this is the same as writing $(x,y) =  (1,2)t+(3,4)$ or similarly  $(x,y)=(1t+3,2t+4)$.)
\begin{enumerate}
	\item Where is the horse at time $t=0,1,2$? 
	\item Draw an set of axis and put markers on your graph to show the horse's location.
	\item Draw the path followed by the horse.
\end{enumerate}

\end{problem}


%Day two of vectors
%\uday
%\large Topical Objectives: \normalsize \\
%\indent $\bullet$ understand vectors as quantities having length and direction, independent of position\\
%\indent $\bullet$ express curves with parametric (vector) equations\\
%\indent $\bullet$ perform addition and scalar multiplication of vectors
\instructor{Approximate end of Unit 1 : Day 1}

In the last two problems we described the donkey and horse's paths using vectors. The use of vectors actually lets us do a lot more like give a generalized direction or easily compute their speed at any point along the path. To do that we need to define two more mathematical ideas, the magnitude and unit-vector.

\begin{definition}
The magnitude, or length, or norm of a vector $\vec v = \left<v_1,v_2,v_3\right>$ is $|\vec v| = \sqrt{v_1^2+v_2^2+v_3^2}$. It is just the distance from the point $(v_1,v_2,v_3)$ to the origin.

Note that in 1D, the length of the vector $\left<-2\right>$ is simply $|-2|=\sqrt{(-2)^2}=2$, the distance to 0. Our use of the absolute value symbols is appropriate, as it generalizes the concept of absolute value (distance to zero) to all dimensions.

\end{definition}

In the special circumstance where a vector represents an object's velocity, the magnitude of the vector gives the objects \textit{speed} or non-directional velocity. 

\begin{definition} A unit vector is a vector whose length is one unit. The notation for a unit vector is generally a ``$\hat{\ }$'' or \textbf{bold face}. Equivalently, the unit vector of a vector $\vec{v}$ is: $\hat{v}=\frac{\vec{v}}{|\vec{v}|}$\\
The standard unit vectors are $\mathbf{i}=\left<1,0,0\right>$, $\mathbf{j}=\left<0,1,0\right>$, $\mathbf{k}=\left<0,0,1\right>$. 
\end{definition}

\begin{problem}[Magnitude and Unit Vector Practice]
For each of the following vectors, compute the magnitude and unit vector in the same direction.
\begin{enumerate}
	\item $-3\ii + 7 \jj$ \instructor{\# 1 Length $\sqrt{58}$}
	\item $<-4,2,4>$ \instructor{\# 2: Length - 6, Unit Vector: $<\frac{-2}{3}, \frac{1}{3}, \frac{2}{3}>$}
	\item $8\ii - \jj + 4 \kk$ \instructor{\# 3 Length 9}
\end{enumerate}
\end{problem}

\begin{problem}
\marginpar{
	\thomasee{See 12.2: 9,17,25,33 and surrounding.} 
	\stewarts{See 12.2: 23-26, 41, 42}
	}%
Consider the two points $P=(1,2,3)$ and $Q=(2,-1,0)$. 
\begin{enumerate}
	\item Write the vector $\vec {PQ}$ in component form $<a,b,c)\>$. 
	\item Find the length (magnitude) of vector $\vec {PQ}$. 
	\item Find a unit vector for $\vec{PQ}$. 
	\item Finally, find a vector of length 7 units that points in the same direction as $\vec{PQ}$. 
\end{enumerate}
\end{problem}

\begin{problem}
Let's return to problems \ref{prob:donkey} and \ref{prob:horseline}. We now have the tools to determine the donkey's and horse's speed and direction.
\begin{enumerate} 
	\item The donkey's path was given by $(x,y)=\vec v t = (3t,-t)$. Find
	\begin{enumerate}
		\item The donkey's speed
		\item A unit vector that gives the donkey's direction of travel.
	\end{enumerate}
	\item The horse's path was given by $\vec r(t) = (1,2)t + (3,4) $.
	\begin{enumerate}
		\item State the part of $\vec r(t)$ which gives the direction of travel.
		\item What is a unit vector in the same direction?
		\item what is the speed of the horse?
	\end{enumerate}
\end{enumerate}
\end{problem}


\begin{problem}
\marginpar{
	\thomasee{See 12.5: 1-12.} 
	\stewarts{Look at 12.2:34-40}
	}%
A raccoon is sitting at point $P=(0,2,3)$.  It starts to climb in the direction $\vec v=\left<1,-1,2\right>$.\\
Write a vector equation $(x,y,z)=(?,?,?)$ for the line that passes through the point $P$ and is parallel to $\vec v$. [Hint, study problem \ref{prob:horseline}, and base your work off of what you saw there. It's almost identical.] \\
Generalize your work to give an equation of the line that passes through the point $P=(x_1,y_1,z_1)$ and is parallel to the vector $\vec v=\langle v_1,v_2,v_3 \rangle$. 
\end{problem}

Make sure you ask me in class to show you how to connect the equation developed above to what you have been doing since middle school. If you can remember $y=mx+b$, then you can quickly remember the equation of a line.  If I don't show you in class, make sure you ask me (or feel free to come by early and ask before class).

%\newpage
%\noindent \Large After Class: \normalsize

\begin{problem}\label{first line between two points}%
\marginpar{
	\thomasee{See 12.5: 13-20.} 
	\stewarts{12.5: 6-15}
	}
Let $P=(3,1)$ and $Q=(-1,4)$.  
\begin{enumerate}
\item Write a vector equation $\vec r(t)=(x,y)$ for (i.e, give a parametrization of) the line that passes through $P$ and $Q$, with $\vec r(0)=P$ and $\vec r(1)=Q$.
\item Write a vector equation for the line that passes through $P$ and $Q$, with $\vec r(0)=P$ but whose speed is twice the speed of the first line.
\item Write a vector equation for the line that passes through $P$ and $Q$, with $\vec r(0)=P$ but whose speed is one unit per second.
\end{enumerate}
\end{problem}


%\uday
\section{The Dot Product}\instructor{Approximate Day 3}
\large Topical Objectives: \normalsize\\
\indent $\bullet$ perform the dot-product of two vectors\\
\indent $\bullet$ recognize when two vectors are orthogonal

\vskip0.2in

Now that we've learned how to add and subtract vectors, stretch them by scalars, and use them to find lines, it's time to introduce a way of multiplying vectors called the dot product.  We'll use the dot product to help us find find angles. First, we need to recall the law of cosines.
\begin{theorem*}[The Law of Cosines]
Consider a triangle with side lengths $a$, $b$, and $c$. Let $\theta$ be the angle between the sides of \textbf{length} $a$ and $b$. Then the law of cosines states that 
$$c^2=a^2+b^2-2ab\cos\theta.$$
If $\theta=90^\circ$, then $\cos\theta=0$ and this reduces to the Pythagorean theorem.
\end{theorem*}

\begin{problem}\instructor{Recommended as pre-class problem} 
\marginpar{
	\thomasee{See 12.3: 9-12.} 
	\stewarts{See 12.3:15-20}
	}%
Sketch in $\mathbb{R}^2$ the vectors $\left<1,2\right>$ and $\left<3,5\right>$.  Use the law of cosines to find the angle between the vectors.
\end{problem}

\begin{problem}\label{prob:dot angle practice}  \instructor{Recommended as pre-class problem}
\marginpar{
	\thomasee{See 12.3: 9-12.} 
	\stewarts{See 12.3:15-20}
	}%
Sketch in $\mathbb{R}^3$ the vectors $\left<1,2,3\right>$ and $\left<-2,1,0\right>$.  Use the law of cosines to find the angle between the vectors.
\end{problem}

\begin{definition}[The Dot Product]
If $\vec u = (u_1,u_2,u_3)$ and $\vec v= (v_1,v_2,v_3)$ are vectors in $\mathbb{R}^3$, then we define the dot product of these two vectors to be 
$$\vec u\cdot \vec v = u_1 v_1+ u_2 v_2+ u_3 v_3.$$
A similar definition holds for vectors in $\mathbb{R}^n$, where
$\vec u\cdot \vec v = u_1 v_1+ u_2 v_2+\cdots+ u_n v_n.$
You just multiply corresponding components together and then add. It is the same process used in matrix multiplication.
\end{definition}

\begin{problem}\label{prob:dot angle practice2}\instructor{Recommended as pre-class problem}
\begin{enumerate}
	\item Use the formula $\vec u\cdot \vec v=|\vec u||\vec v|\cos\theta$ to find the angle between the vectors $\left<1,2,3\right>$ and $\left<-2,1,0\right>$.
	\item Which was easier, \ref{prob:dot angle practice} or this method?  (You will derive this formula in a later problem)
\end{enumerate}
\end{problem}

\begin{definition}\label{def:orthogonal}
We say that the vectors $\vec u$ and $\vec v$ are orthogonal if $\vec u\cdot \vec v=0$. 
\end{definition}

\begin{problem}\instructor{Recommended as pre-class problem}
Find two vectors orthogonal to $(1,2)$.  Then find 4 vectors orthogonal to $(3,2,1)$.  
\end{problem}

\begin{problem}\label{dot product facts}
Mark each statement true or false. Use Definitions \ref{def:vecadd} - \ref{def:orthogonal} to explain and justify or prove your claim. You can assume that $\vec u,\vec v,\vec w\in\mathbb{R}^3$ and that $c\in\mathbb{R}$.
\begin{enumerate}
\item $\vec u\cdot \vec v=\vec v\cdot \vec u$. 
\item $\vec u\cdot (\vec v\cdot \vec w)=(\vec u\cdot\vec v)\cdot\vec w$. 
\item $c(\vec u\cdot \vec v)=(c\vec u)\cdot \vec v=\vec u\cdot (c\vec v)$. 
\item $\vec u+(\vec v\cdot \vec w)=(\vec u+\vec v)\cdot(\vec u+\vec w)$. 
\item $\vec u\cdot (\vec v+ \vec w)=(\vec u\cdot \vec v)+(\vec u\cdot\vec w)$. 
\item $\vec u\cdot \vec u= |\vec u|^2$. 
\end{enumerate}
\end{problem}

%\newpage
%\noindent \Large After Class: \normalsize

\begin{problem}\label{prob:dot prep} 
\marginpar{
	\thomasee{Page 693 has the solution if you are struggling.} 
	}
If $\vec u = (u_1,u_2,u_3)$ and $\vec v= (v_1,v_2,v_3)$ are vectors in $\mathbb{R}^3$ (which is often written $\vec u,\vec v\in\mathbb{R}^3$), then show that 
$$|\vec u-\vec v|^2 = |\vec u|^2-2\vec u\cdot \vec v +|\vec v|^2.$$
\end{problem}

\begin{problem}\label{prob:dot angle formula}  
\marginpar{
	\thomasee{See page 693.} 
	\stewarts{See pages 801-802 if you are struggling}
	}%
Let $\vec u,\vec v\in\mathbb{R}^3$. Let $\theta$ be the angle between $\vec u$ and $\vec v$. 
\begin{enumerate}
\item Use the law of cosines to explain why $|\vec u-\vec v|^2=|\vec u|^2+|\vec v|^2-2|\vec u||\vec v|\cos\theta$.
\item Use the above together with problem \ref{prob:dot prep} to derive $$\vec u\cdot \vec v=|\vec u||\vec v|\cos\theta.$$
\end{enumerate}
\end{problem}

\begin{problem} 
\marginpar{
	\thomasee{See page 694.} 
	\stewarts{See page 804}
	}%
Show that if two nonzero vectors $\vec u$ and $\vec v$ are orthogonal, then the angle between them is 90$^\circ$. Then show that if the angle between them is 90$^\circ$, then the vectors are orthogonal. I.E. expand and compute both sides of the formula $\vec u\cdot \vec v=|\vec u||\vec v|\cos\theta$ with non-zero, orthogonal vectors.
\end{problem}

The dot product provides a really easy way to find when two vectors meet at a right angle. The dot product is precisely zero when this happens.

% \begin{problem}
%   Show that the distance from a point $Q$ to a line (with direction vector $\vec v$ passing through $P$) is $|\overrightarrow{PQ}-\proj_{\vec v}\overrightarrow {PQ}|$. Draw a diagram illustrating your reasoning.
% \end{problem}

%\uday

%=========================================================================================================================
%When Valpo is on, include the Matrix Introduction/Review here.
%===================================================================================================

\ifvalpo
\section{Interlude:Matrices}\label{review:matrices}
We will soon discover that matrices represent derivatives in high dimensions. When you use matrices to represent derivatives, the chain rule is precisely matrix multiplication. For now, we just need to become comfortable with matrix multiplication.

We perform matrix multiplication ``row by column''.  Wikipedia has an excellent visual illustration of how to do this. See \marginpar{ \bmw{The electronic version has links that will open your browser and take you to the web.} \valpo{The electronic version has links that will open your browser and take you to the web.}}
\href{http://en.wikipedia.org/wiki/Matrix\_multiplication}{Wikipedia} for an explanation. Alternatively see \href{http://www.texample.net/tikz/examples/matrix-multiplication/}{texample.net} for a visualization of the idea.

\begin{problem} 
	\marginpar{ 
	\bmw{For extra practice, make up two small matrices and multiply them.  Use 
\href{http://aleph.sagemath.org/?z=eJxztM1NLCnKrNCIjjbUMdYxiY3V5HJCiJnrGMXqKICkQJSukY4BSIGjlhMA16EPQw}{Sage}
or
\href{http://www.wolframalpha.com/input/?i=\%281\%2C3\%2C4\%29+*\%28\%287\%2C2\%29\%2C\%281\%2C3\%29\%2C\%28-2\%2C0\%29\%29}{Wolfram
  Alpha} to see if you are correct (click the links to see how to do
matrix multiplication in each system).}
	\valpo{For extra practice, make up two small matrices and multiply them.  Use 
\href{http://aleph.sagemath.org/?z=eJxztM1NLCnKrNCIjjbUMdYxiY3V5HJCiJnrGMXqKICkQJSukY4BSIGjlhMA16EPQw}{Sage}
or
\href{http://www.wolframalpha.com/input/?i=\%281\%2C3\%2C4\%29+*\%28\%287\%2C2\%29\%2C\%281\%2C3\%29\%2C\%28-2\%2C0\%29\%29}{Wolfram
  Alpha} to see if you are correct (click the links to see how to do
matrix multiplication in each system).}
}% end marginnote
Compute the following matrix products.
\begin{itemize}
\item $\begin{bmatrix}
3 & 2& 1
\end{bmatrix}
\begin{bmatrix}
-1 \\
 2\\
 0
\end{bmatrix}$
\item
$\begin{bmatrix}1 &2\\3&4\end{bmatrix}\begin{bmatrix}5&0\\6&1\end{bmatrix}$
\end{itemize} \end{problem}


\begin{problem} Compute the product
$\begin{bmatrix}
3 & 2& 1\\
0 & 1& -4
\end{bmatrix}
\begin{bmatrix}
-1&3 &0 \\
 2&-1 &0\\
 0&1 &2
\end{bmatrix}$.
\end{problem}


\subsection{Determinants}
\label{sec:Determinants}

Associated with every square matrix is a number, called the determinant.  Determinants are only defined for square matrices.
Determinants measure area, volume, length, and higher dimensional versions of these ideas.  Determinants will appear as we study cross products and when we get to the high dimensional version of {$u$}-substitution.
\begin{definition}
The determinant of a {$2\times 2$} matrix is the number 
	\marginpar{We use vertical bars next to a matrix to state we want the determinant, so $\det A = |A|$. } 
\begin{align*}
\det\begin{bmatrix}a&b\\c&d\end{bmatrix} &=\begin{vmatrix}a&b\\c&d\end{vmatrix} = ad-bc.
\end{align*}
The determinant of a {$3\times 3$} matrix is the number 
	\marginpar{Notice the negative sign on the middle term of the {$3 \times 3$} determinant. Also, notice that we had to compute three determinants of 2 by 2 matrices in order to find the determinant of a 3 by 3.} 
\begin{align*}
\begin{vmatrix}a&b&c\\d&e&f\\g&h&i\end{vmatrix} &= a\det\begin{vmatrix}e&f\\h&i\end{vmatrix} -b\det\begin{vmatrix}d&f\\g&i\end{vmatrix} +c\det\begin{vmatrix}d&e\\g&h\end{vmatrix}\\
&=a(ei-hf)-b(di-gf)+c(dh-ge).
\end{align*}
\end{definition}

\begin{problem}
	\marginpar{For extra practice, create your own square matrix (2 by 2 or 3 by 3) and compute the determinant by hand. Then use \href{http://www.wolframalpha.com}{Wolfram Alpha} to check your work.  Do this until you feel comfortable taking determinants.}
Compute 
$\begin{vmatrix}
1&2\\
3&4
\end{vmatrix} 
$
and 
$\begin{vmatrix}
1&2&0\\
-1&3&4\\
2&-3&1
\end{vmatrix} 
$.
\end{problem}

What good is the determinant?  
The determinant was discovered as a result of trying to find the area of a parallelogram and the volume of the three dimensional version of a parallelogram (called a parallelepiped) in space. 
If we had a full semester to spend on linear algebra, we could eventually prove the following facts that I will just present here with a few examples.

Consider the 2 by 2 matrix $\begin{bmatrix}3&1\\0&2\end{bmatrix}$ whose determinant is $3\cdot 2-0\cdot 1=6$. Draw the column vectors $\begin{bmatrix}3\\0\end{bmatrix}$ and $\begin{bmatrix}1\\2\end{bmatrix}$ with their base at the origin (see figure \ref{detfig}). 
These two vectors give the edges of a parallelogram whose area is the determinant $6$.  If I swap the order of the two vectors in the matrix, then the determinant of $\begin{bmatrix}1&3\\2&0\end{bmatrix}$ is $-6$.  The reason for the difference is that the determinant not only keeps track of area, but also order. Starting at the first vector, if you can turn counterclockwise through an angle smaller than 180$^\circ$ to obtain the second vector, then the determinant is positive.  If you have to turn clockwise instead, then the determinant is negative.  This is often termed ``the right-hand rule,'' as rotating the fingers of your right hand from the first vector to the second vector will cause your thumb to point up precisely when the determinant is positive.
%\marginpar{{
\begin{figure}[h]
\begin{center}
\begin{tikzpicture}[scale=.8]
\draw[help lines,step=1cm] (0,0) grid (4,2);
\draw[->,>=stealth,red] (0,0) -- (3,0);
\draw[->,>=stealth,red] [shift={(1,2)}](0,0) -- (3,0);
\draw[->,>=stealth,blue] (0,0) -- (1,2);
\draw[->,>=stealth,blue] [shift={(3,0)}](0,0) -- (1,2);
\draw[->,>=stealth] (0:1cm)  node[above right=1pt,fill=white]{\normalsize $+$} arc (0:64:1cm) ;
\draw[<-,>=stealth] (0:2cm)  node[above right=1pt,fill=white]{\normalsize $-$} arc (0:64:2cm) ;
\node[fill=white] at (2.5, 1.5) {Area $=6$}; 
\end{tikzpicture}

\vspace{2pt}
$\begin{vmatrix}{3}&{1}\\{0}&{2}\end{vmatrix}=6$ and $\begin{vmatrix}{1}&{3}\\{2}&0\end{vmatrix}=-6$
\end{center}
\caption{The determinant gives both area and direction. A counter clockwise rotation from column 1 to column 2 gives a positive determinant.\label{detfig}}
\end{figure}
%    }}

For a 3 by 3 matrix, the columns give the edges of a three dimensional parallelepiped and the determinant produces the volume of this object. The sign of the determinant is related to orientation. If you can use your right hand and place your index finger on the first vector, middle finger on the second vector, and thumb on the third vector, then the determinant is positive. 
For example, consider the matrix $A = \begin{bmatrix}\cl{1\\0\\0}&\cl{0\\2\\0}&\cl{0\\0\\3}\end{bmatrix}$.  Starting from the origin, each column represents an edge of the rectangular box 
$0\leq x\leq 1$, 
$0\leq y\leq 2$, 
$0\leq z\leq 3$ with volume (and determinant) $V=lwh=(1)(2)(3)=6$. The sign of the determinant is positive because if you place your index finger pointing in the direction (1,0,0) and your middle finger in the direction (0,2,0), then your thumb points upwards in the direction (0,0,3). 
If you interchange two of the columns, for example 
$B = \begin{bmatrix} \cl{0\\2\\0}&\cl{1\\0\\0}&\cl{0\\0\\3}\end{bmatrix}$, then the volume doesn't change since the shape is still the same. However, the sign of the determinant is negative because if you point your index finger in the direction (0,2,0) and your middle finger in the direction (1,0,0), then your thumb points down in the direction (0,0,-3). If you repeat this with your left hand instead of right hand, then your thumb points up.

\begin{problem}
\begin{enumerate}
\item Use determinants to find the area of the triangle with vertices $(0,0)$, $(-2,5)$, and $(3,4)$.
\item What would you change if you wanted to find the area of the triangle with vertices $(-3,1)$, $(-2,5)$, and $(3,4)$? Find this area.
\end{enumerate}
\end{problem}

\fi
%=========================================================
%End Matrix section that is printed if Valpo is on.
%=========================================================

\section{The Cross Product}
\large Topical Objective: \normalsize \\
\indent $\bullet$ perform the cross-product of vectors\\

\vskip0.2in

The dot product gave us a way of multiplying two vectors together, but the result was a number, not a vectors. We now define the cross product, which will allow us to multiply two vectors together to give us another vector.  We were able to define the dot product in all dimensions.  The cross product is only defined in $\mathbb{R}^3$. 

\begin{definition}[The Cross Product]
\marginpar{This definition is not really a definition.  It is actually a theorem.  If you use the formula given as the definition, then you would need to prove the three facts. We have the tools to give a complete proof of (1) and (3), but we would need a course in linear algebra to prove (2). It shouldn't be too much of a surprise that the cross product is related to area, since it is defined in terms of determinants}
The cross product of two vectors $\vec u = \left<u_1,u_2,u_3\right>$ and $\vec v = \left<v_1,v_2,v_3\right>$ is a new vector $\vec u\times \vec v$. This new vector is (1) orthogonal to both $\vec u$ and $\vec v$, (2) has a length equal to the area of the parallelogram whose sides are these two vectors, and (3) points in the direction your thumb points as you curl the base of your right hand from $\vec u$ to $\vec v$. The formula for the cross product is $$\vec u\times \vec v = \left<u_2v_3-u_3v_2,-(u_1v_3-u_3v_1),u_1v_2-u_2v_1\right> = \det\begin{bmatrix}\vec i & \vec j&\vec k\\ u_1&u_2&u_3\\ v_1&v_2&v_3\\\end{bmatrix}.$$
\end{definition}

\begin{problem} \instructor{Recommended as Pre-class} 
\marginpar{
	\thomasee{See 12.4: 1-8.} 
	\stewarts{See 12.4:1-7}
	}%
Let $\vec u=\langle 1,-2,3\rangle $ and $\vec v=\langle 2,0,-1\rangle$.  
\begin{enumerate}
\item Compute $\vec u\times \vec v$ and $\vec v\times \vec u$.  How are they related?
\item Compute $\vec u \cdot (\vec u\times \vec v)$ and $\vec v \cdot (\vec u\times \vec v)$. Why did you get the answer you got?
\item Compute $\vec u \times (2\vec u)$.  Why did you get the answer you got?
%\item Compute $|\vec u \times \vec v|$.  Compute the area of the parallelogram formed by $\vec u$ and $\vec v$ using trigonometry and $|\vec u|$, $|\vec v|$, and the angle $\theta$ between the two vectors. Compare your answer with $|\vec u \times \vec v|$.
\end{enumerate}
\end{problem}

\begin{problem} \instructor{Recommended as Pre-class}
\marginpar{
	\thomasee{See 12.4: 9-14.} 
	}%
Consider the vectors ${\ii}=(1,0,0)$, ${2\jj}=(0,2,0)$, and ${3\kk}=(0,0,3)$.
\begin{enumerate}
\item Compute $\ii\times {2\jj}$ and ${2\jj}\times {\ii}$.
\item Compute ${\ii}\times {3\kk}$ and ${3\kk}\times {\ii}$.
\item Compute ${2\jj}\times {3\kk}$ and ${3\kk}\times {2\jj}$.
\end{enumerate}
Give a geometric reason as to why some vectors above have a plus sign, and some have a minus sign.
\end{problem}

\begin{problem}  \instructor{Recommended as Pre-class}
\marginpar{
	\thomasee{See 12.4: 15-18.} 
	\stewarts{See 12.4: 29-32}
	}%
Let $P=(2,0,0)$, $Q=(0,3,0)$, and $R=(0,0,4)$. Find a vector that orthogonal to both $\vec {PQ}$ and $\vec {PR}$. Then find the area of the triangle $PQR$ [Hint: What shape does two triangles side-by-side make?]. Construct a 3D graph of this triangle. 
\end{problem}

\section{The Cross Product and Planes}
\large Topical Objectives \normalsize \\
\indent $\bullet$ use the normal vector to find the equation for a plane

\vskip0.2in

We will now combine the dot product with the cross product to develop an equation of a plane in 3D. 
Before doing so, let's look at what information we need to obtain a line in 2D, and a plane in 3D.  
To obtain a line in 2D, one way is to have 2 points. 
The next problem introduces the new idea by showing you how to find an equation of a line in 2D. 

\begin{problem}\label{prob:plane equation normal point}
Suppose the point $P=(1,2)$ lies on line $L$. Suppose that the angle between the line and the vector $\vec n=\left<3,4\right>$ is 90$^\circ$ (whenever this happens we say the vector $\vec n$ is normal to the line). Let $Q=(x,y)$ be another point on the line $L$. Use the fact that $\vec n$ is orthogonal to $\vec {PQ}$ to obtain an equation of the line $L$. 
\end{problem}

\note{Add a problem about the equation of a plane containing two vectors.}

\begin{problem}\label{plane equation three points}
\marginpar{
	\thomasee{See page 709.}
	\larsonfive{See Larson 11.5.} 
	\stewarts{See pages 819-822}
	}%
Let $P=(a,b,c)$ be a point on a plane in 3D. Let $\vec n=\langle A,B,C \rangle $ be a normal vector to the plane (so the angle between the plane and and $\vec n$ is 90$^\circ$).  Let $Q=(x,y,z)$ be another point on the plane.  Show that an equation of the plane through point $P$ with normal vector $\vec n$ is $$A(x-a)+B(y-b)+C(z-c)=0.$$
\end{problem}

%\newpage
%\noindent \Large After Class: \normalsize


\begin{problem}\label{plane equation 2 lines}
\marginpar{
	\larsonfive{See Larson 11.5:47--58 for more practice.}
	}%
Find an equation of the plane containing the lines $\vec r_1(t)=(1,3,0)t+(1,0,2)$ and $\vec r_2(t)=(2,0,-1)t+(2,3,2)$.
\end{problem}

\begin{problem}  
\marginpar{
	\thomasee{See 12.5: 21-28.}
	\larsonfive{See Larson 11.5:47--58 for more practice.} 
	\stewarts{See 12.5: 23-40}
	}%
Consider the three points $P=(1,0,0), Q=(2,0,-1), R=(0,1,3)$. Find an equation of the plane which passes through these three points.  [Hint: First find a normal vector to the plane.]
\end{problem}

\begin{problem}\label{prob:crossproduct normalvector}
 Consider the points $P=(2,-1,0)$, $Q=(0,2,3)$, and $R=(-1,2,-4)$.  
\begin{enumerate}
 \item Give an equation $(x,y,z)=(?,?,?)$ of the line through $P$ and $Q$.
 \item Give an equation of the line through $P$ and $R$.
 \item Give an equation of the plane through $P$, $Q$, and $R$. 
\end{enumerate}
\end{problem}

\begin{problem}  
\marginpar{
	\thomasee{See 12.5: 57-60.}
	\larsonfive{See Larson 11.5:91--92 for more practice.} 
	\stewarts{See 12.5: 23-40, 45-47}
	}%
Consider the two planes $x+2y+3z=4$ and $2x-y+z=0$.  These planes meet in a line.  Find a vector that is parallel to this line.  Then find a vector equation of the line.
\end{problem}


%\uday
\section{Projections and Their Applications}
\large Course Objective: \normalsize \\
\indent $\bullet$ See applications of multi-variable calculus

\vskip0.2in

Suppose a heavy box needs to be lowered down a ramp.  
The box exerts a downward force of 200 Newtons, which we will write in vector notation as $\vec F=\left<0,-200\right>$. 
The ramp was placed so that the box needs to be moved right 6 m, and down 3 m, so we need to get from the origin $(0,0)$ to the point $(6,-3)$.  This displacement can be written as $\vec d=\left<6,-3\right>$. The force $F$ acts straight down, which means the ramp takes some of the force. Our goal is to find out how much of the 200N the ramp takes, and how much force must be applied to prevent the box from sliding down the ramp (neglecting friction). We are going to break the force $\vec F$ into two components, one component in the direction of $\vec d$, and another component orthogonal to $\vec d$. 

\begin{problem}\label{prob:force intro}
Read the preceding paragraph.\\
We want to write $\vec F$ as the sum of two vectors: $\vec F = \vec w+\vec n$
\begin{itemize}
	\item where $\vec w$ is parallel to $\vec d$ 
	\item and $\vec n$ is orthogonal to $\vec d$
\end{itemize}
Since $\vec w$ is parallel to $\vec d$, we can write $\vec w=c\vec d$ for some unknown scalar $c$.\\
\begin{enumerate}
	\item Rewrite $\vec F$ in terms of $\vec d$
	\item Take the dot-product of both sides with $\vec d$
	\item Since $\vec n$ is orthogonal to $\vec d$ we know that $\vec n \cdot \vec d =$ ?
	\item Substitute and solve for the unknown $c$
\end{enumerate}
\end{problem}

The solution to the previous problem gives us the definition of a projection.

\begin{definition}
The projection of $\vec F$ onto $\vec d$, written $\proj_{\vec d}\vec F$, is defined as $$\proj_{\vec d}\vec F = \left(\frac{\vec F\cdot \vec d}{\vec d\cdot \vec d}\right)\vec d.$$
\end{definition}

\hrule

\begin{problem} 
\marginpar{
	\thomasee{See 12.3:1-8 (part d).} 
	\stewarts{See 12.3: 39-44}
	}%
Let $\vec u=(-1,2)$ and $\vec v=(3,4)$. Compute the $\proj_{\vec v}\vec u$. Draw $\vec u,$ $\vec v$, and $\proj_{\vec v}\vec u$ all on one set of axis. Then draw a line segment from the head of $\vec u$ to the head of the projection.\\

Now let $\vec w=(-2,0)$. Compute $\proj_{\vec v}\vec w$. Draw $\vec u,$ $\vec v$, and $\proj_{\vec v}\vec w$. Then draw a line segment from the head of $\vec w$ to the head of the projection.

\end{problem}

One application of projections pertains to the concept of work. Work is the transfer of energy. If a force $F$ acts through a displacement $d$, then the most basic definition of work is $W=Fd$, the product of the force and the displacement.  This basic definition has a few assumptions.
\begin{itemize}
\item The force $F$ must act in the same direction as the displacement.
\item The force $F$ must be constant throughout the entire displacement.
\item The displacement must be in a straight line.
\end{itemize}
Before the semester ends, we will be able to remove all 3 of these assumptions.  The next problem will show you how dot products help us remove the first assumption.

Recall the set up to problem \ref{prob:force intro}.  We want to lower a box down a ramp (which we will assume is frictionless). Gravity exerts a force of $\vec F=\left<0,-200\right>$ N. If we apply no other forces to this system, then gravity will do work on the box through a displacement of $\left<6,-3\right>$ m. The work done by gravity will transfer the potential energy of the box into kinetic energy (remember that work is a transfer of energy).  How much energy is transferred?

\begin{problem}[Projection Application: Work]\label{first work problem}
\marginpar{
	\thomasee{See 12.3: 24, 41-44.} 
	\stewarts{See 12.3:49-52}
	}% 
Find the amount of work done by the force $\vec F=\left<0,-200\right>$ through the displacement $\vec d=\left<6,-3\right>$. Find this by doing the following:
\begin{enumerate}
\item Find the projection of $\vec F$ onto $\vec d$. This tells you how much force acts in the direction of the displacement. Find the magnitude of this projection.
\item Since work equals $W=Fd$, multiply your answer above by $|\vec {d}|$.  
\item Now compute $\vec F\cdot \vec d$. You have just shown that $W=\vec F\cdot \vec d$ when $\vec F$ and $\vec d$ are not in the same direction.
\end{enumerate}
\end{problem}

%\newpage
%\noindent \Large After Class: \normalsize
%\section{More Planes}

\begin{problem}[Projection Application: Planes] 
\marginpar{
	\stewarts{See 12.5:69-72}
	}%
 Consider the points $P=(2,4,5)$, $Q=(1,5,7)$, and $R=(-1,6,8)$.
\begin{enumerate}
 \item What is the area of the triangle $PQR$. 
 \item Give a normal vector to the plane through these three points.
 \item What is the distance from the point $A=(1,2,3)$ to the plane $PQR$.  [Hint: Compute the projection of $\vec {PA}$ onto $\vec n$.  How long is it?] 
\end{enumerate}

 
\end{problem}



% \begin{problem}
%   Show that the distance from a point $Q$ to a plane (with normal vector {$\vec n$} and a point $P$) is given by $|\proj_{\vec n}\overrightarrow {PQ}|$. Draw a diagram illustrating your reasoning.
% \end{problem}

% \begin{problem}
%   Show that the distance from a line (with direction vector $\vec v_1$ passing through $P_1$) to a line (with direction vector $\vec v_2$ passing through $P_2$) is $|\proj_{\vec v_1\times\vec v_2}\overrightarrow {P_1P_2}|$. Draw a diagram illustrating your reasoning.
% \end{problem}

\note{Here are two more problems about dot and cross product, but they aren't central to the course.  Maybe these could be good problems for advanced students.  I saw these in Larson:

  \begin{problem} If the statement is true, explain why.  If it is false, give a counterexample.
    \begin{enumerate}
    \item If $\vec u \neq 0$ and $\vec u \times \vec v = \vec u \times \vec w$, then is $\vec v = \vec w$ always?
    \item If $\vec u \neq 0$, $\vec u\cdot \vec v = \vec u \cdot \vec w$,  and $\vec u \times \vec v = \vec u \times \vec w$, then is $\vec v = \vec w$ always?
    \end{enumerate}
  \end{problem}
}

\clearpage

\wrapup
\clearpage
%
%

\note{In the Latex Main file, there should probably be an option for turning on Ben's original chapters here instead of Karl's sliced and diced chapters}

\chapter{Review: Conic Sections}
\minitoc \mtcskip
\noindent 
Upon completing this unit you will be prepared to...
\begin{enumerate}

\item Describe, graph, give equations of, and find foci for conic sections (parabolas, ellipses, hyperbolas). 


\end{enumerate}



\section{Conic Sections}
Before we jump fully into $\mathbb{R}^3$, we need some good examples of planar curves (curves in $\mathbb{R}^2$) that we'll extend to object in 3D.  These examples are conic sections. We call them conic sections because you can obtain each one by intersecting a cone and a plane (I'll show you in class how to do this).  Here's a definition.

\begin{definition}
Consider two identical, infinitely tall, right circular cones placed
vertex to vertex so that they share the same axis of symmetry.  A conic
section is the intersection of this three dimensional surface with any plane that does
not pass through the vertex where the two cones meet.
\end{definition}

These intersections are called circles (when the plane is perpendicular to the axis of symmetry),
parabolas (when the plane is parallel to one side of one cone), hyperbolas (when the plane
is parallel to the axis of symmetry), and ellipses (when the plane does not meet any of the
three previous criteria). 

The definition above provides a geometric description of how to obtain a conic section from cone.  We'll not introduce an alternate definition based on distances between points and lines, or between points and points.  Let's start with one you are familiar with.

\begin{definition}
Consider the point $P=(a,b)$ and a positive number $r.$ A circle 
circle with center $(a,b)$ and radius $r$ is
the set of all points $Q=(x,y)$ in the plane so that the segment $PQ$ has length $r$. 
\end{definition}

Using the distance formula, this means that every circle can be written in the form $(x-a)^2+(y-b)^2=r^2$. 

\begin{problem} 
The equation $4x^2+4y^2+6x-8y-1=0$ represents a circle (though initially it does not look like it). Use the method of completing the square to rewrite the equation in the form
$(x-a)^2 + (y-b)^2 = r^2$ (hence telling you the center and radius). Then generalize
your work to find the center and radius of any circle written in the form $x^2+y^2+Dx+Ey+F=0$.
\end{problem}

\subsection{Parabolas}
Before proceeding to parabolas, we need to define the distance between a point and a line.

\begin{definition}
Let $P$ be a point and $L$ be a line.  Define the distance between $P$ and $L$ (written
$d(P,L)$) to be the length of the shortest line segment that has one end on $L$ and the other end on $P$. Note: This segment will always be perpendicular to $L$.
\end{definition}

\begin{definition}
Given a point $P$ (called the focus) and a line $L$ (called the directrix) which does not pass through $P$, we define a parabola as the set of all points $Q$ in the plane so that the distance from $P$ to $Q$ equals the distance from $Q$ to $L$. 
The vertex is the point on the parabola that is closest to the directrix.
\end{definition}

\begin{problem}  
\marginpar{
	\thomasee{See page 658.}
	}%
Consider the line $L:y=-p$, the point $P=(0,p)$, and another point $Q=(x,y)$.  Use the distance formula to show that an equation of a parabola with directrix $L$ and focus $P$ is $x^2=4py$.
Then use your work to explain why an equation of a parabola with directrix $x=-p$ and focus $(p,0)$ is $y^2=4px$. 
\end{problem}

Ask me about the reflective properties of parabolas in class, if I have not told you already.  They are used in satellite dishes, long range telescopes, solar ovens, and more.  The following problem provides the basis to these reflective properties and is optional.  If you wish to present it, let me know. I'll have you type it up prior to presenting in class.

\begin{problem*}[Optional]
Consider the parabola $x^2=4py$ with directrix $y=-p$ and focus $(0,p)$. Let $Q=(a,b)$ be some point on the parabola. Let $T$ be the tangent line to $L$ at point $Q$. Show that the angle between $PQ$ and $T$ is the same as the angle between the line $x=a$ and $T$. This shows that a vertical ray coming down towards the parabola will reflect of the wall of a parabola and head straight towards the vertex.    
\end{problem*}

The next two problems will help you use the basic equations of a parabola, together with shifting and reflecting, to study all parabolas whose axis of symmetry is parallel to either the $x$ or $y$ axis. 

\begin{problem} 
\marginpar{
	\thomasee{See 11.6: 9-14}
	}%
Once the directrix and focus are known, we can give an equation of a parabola. For each of the following, give an equation of the parabola with the stated directrix and focus. Provide a sketch of each parabola.
\begin{enumerate}
\item The focus is $(0,3)$ and the directrix is $y=-3$.
\item The focus is $(0,3)$ and the directrix is $y=1$.
\end{enumerate}
\end{problem}

\begin{problem}
Give an equation of each parabola with the stated directrix and focus. Provide a sketch of each parabola.
\begin{enumerate}
\item The focus is $(2,-5)$ and the directrix is $y=3$.
\item The focus is $(1,2)$ and the directrix is $x=3$.
\end{enumerate}
\end{problem}

\begin{problem}  
\marginpar{
	\thomasee{See 11.6: 9-14}
	}%
Each equation below represents a parabola.  Find the focus, directrix, and vertex of each parabola, and then provide a rough sketch.
\begin{enumerate}
\item $y=x^2$
\item $(y-2)^2=4(x-1)$
\end{enumerate}
\end{problem}

\begin{problem}
Each equation below represents a parabola.  Find the focus, directrix, and vertex of each parabola, and then provide a rough sketch.
\begin{enumerate}
\item $y=-8x^2+3$
\item $y=x^2-4x+5$
\end{enumerate}
\end{problem}

\subsection{Ellipses}

\begin{definition}
Given two points $F_1$ and $F_2$ (called foci) and a fixed distance $d$, we define an ellipse as the set of all points $Q$ in the plane so that the sum of the distances $F_1Q$  and $F_2Q$ equals the fixed distance $d$. The center of the ellipse is the midpoint of the segment $F_1F_2$. The two foci define a line.  Each of the two points on the ellipse that intersect this line is called a vertex. The major axis is the segment between the two vertexes. The minor axis is the largest segment perpendicular to the major axis that fits inside the ellipse.
\end{definition}

We can derive an equation of an ellipse in a manner very similar to how we obtained an equation of a parabola.  The following problem will walk you through this.  \bmw{We will not have time to present this problem in class. However, if you would like to complete the problem and write up your solution on the wiki, you can obtain presentation points for doing so.  Let me know if you are interested. }

\begin{problem*}[Optional]
Consider the ellipse produced by the fixed distance $d$ and the foci $F_1=(c,0)$ and $F_2=(-c,0)$. Let $(a,0)$ and $(-a,0)$ be the vertexes of the ellipse.
\begin{enumerate}
\item Show that $d=2a$ by considering the distances from $F_1$ and $F_2$ to the point $Q=(a,0)$.
\item Let $Q=(0,b)$ be a point on the ellipse.  Show that $b^2+c^2=a^2$ by considering the distance between $Q$ and each focus.
\item Let $Q=(x,y)$. By considering the distances between $Q$ and the foci, show that an equation of the ellipse is $$\frac{x^2}{a^2}+\frac{y^2}{b^2}=1.$$
\item Suppose the foci are along the $y$-axis (at $(0,\pm c)$) and the fixed distance $d$ is now $d=2b$, with vertexes $(0,\pm b)$. Let $(a,0)$ be a point on the $x$ axis that intersect the ellipse.  Show that we still have $$\frac{x^2}{a^2}+\frac{y^2}{b^2}=1,$$ but now we instead have $a^2+c^2=b^2$.
\end{enumerate}
\end{problem*}

You'll want to use the results of the previous problem to complete the problems below. The key equation above is $\frac{x^2}{a^2}+\frac{y^2}{b^2}=1$. The foci will be on the $x$-axis if $a>b$, and will be on the $y$-axis if $b>a$. The second part of the problem above shows that the distance from the center of the ellipse to the vertex is equal to the hypotenuse of a right triangle whose legs go from the center to a focus, and from the center to an end point of the minor axis. 

The next three problems will help you use the basic equations of an ellipse, together with shifting and reflecting, to study all ellipses whose major axis is parallel to either the $x$- or $y$-axis. 

\begin{problem}  
\marginpar{
	\thomasee{See 11.6: 17-24}
	}%
For each ellipse below, graph the ellipse and give the coordinates of the foci and vertexes. \begin{enumerate}
\item $\ds 16x^2+25y^2=400$ [Hint: Divide by 400.]
\item $\ds \frac{(x-1)^2}{5}+\frac{(y-2)^2}{9}=1$
\end{enumerate}
\end{problem}

\begin{problem}
For the ellipse $x^2+2x+2y^2-8y=9$, sketch a graph and give the coordinates of the foci and vertexes. 
\end{problem}

\begin{problem} 
\marginpar{
	\thomasee{See 11.6: 25-26}
	}%
Given an equation of each ellipse described below, and provide a rough sketch.
\begin{enumerate}
\item The foci are at $(2\pm 3,1)$ and vertices at $(2\pm 5, 1)$.
\item The foci are at $(-1,3\pm 2)$ and vertices at $(-1, 3\pm 5)$.
\end{enumerate}
\end{problem}

\bmw{Ask me about the reflective properties of an ellipse in class, if I have not told you already. The following problem provides the basis to these reflective properties and is optional.  If you wish to present it, let me know. I'll have you type it up prior to presenting in class.}

\begin{problem*}[Optional]
Consider the ellipse $\frac{x^2}{a^2}+\frac{y^2}{b^2}=1$ with foci $F_1=(c,0)$ and $F_2=(-c,0)$. 
Let $Q=(x,y)$ be some point on the ellipse. 
Let $T$ be the tangent line to the ellipse at point $Q$. 
Show that the angle between $F_1Q$ and $T$ is the same as the angle between $F_2Q$ and $T$. This shows that a ray from $F_1$ to $Q$ will reflect off the wall of the ellipse at $Q$ and head straight towards the other focus $F_2$.
\end{problem*}


\subsection{Hyperbolas}

\begin{definition}
Given two points $F_1$ and $F_2$ (called foci) and a fixed number $d$, we define a hyperbola as the set of all points $Q$ in the plane so that the difference of the distances $F_1Q$  and $F_2Q$ equals the fixed number $d$ or $-d$. The center of the hyperbola is the midpoint of the segment $F_1F_2$. The two foci define a line.  Each of the two points on the hyperbola that intersect this line is called a vertex.
\end{definition}

We can derive an equation of a hyperbola in a manner very similar to how we obtained an equation of an ellipse. The following problem will walk you through this.  \bmw{We will not have time to present this problem in class.}

\begin{problem*}[Optional]
Consider the hyperbola produced by the fixed number $d$ and the foci $F_1=(c,0)$ and $F_2=(-c,0)$. Let $(a,0)$ and $(-a,0)$ be the vertexes of the hyperbola.
\begin{enumerate}
\item Show that $d=2a$ by considering the difference of the distances from $F_1$ and $F_2$ to the vertex $(a,0)$.
\item Let $Q=(x,y)$ be a point on the hyperbola. By considering the difference of the distances between $Q$ and the foci, show that an equation of the hyperbola is $\frac{x^2}{a^2}-\frac{y^2}{c^2-a^2}=1,$ or if we let $c^2-a^2=b^2$, then the equation is 
$$\frac{x^2}{a^2}-\frac{y^2}{b^2}=1.$$
\item Suppose the foci are along the $y$-axis (at $(0,\pm c)$) and the number $d$ is now $d=2b$, with vertexes $(0,\pm b)$. Let $a^2=c^2-b^2$. Show that an equation of the hyperbola is $$\frac{y^2}{b^2}-\frac{x^2}{a^2}=1.$$
\end{enumerate}
\end{problem*}

You'll want to use the results of the previous problem to complete the problems below.

\begin{problem} 
\marginpar{
	\thomasee{See 11.6: 27-34}
	}%
Consider the hyperbola $\frac{x^2}{a^2}-\frac{y^2}{b^2}=1.$ Construct a box centered at the origin with corners at $(a, \pm b)$ and $(-a,\pm b)$.  Draw lines through the diagonals of this box. Rewrite the equation of the hyperbola by solving for $y$ and then factoring to show that as $x$ gets large, the hyperbola gets really close to the lines $y=\pm \frac{b}{a}x$. [Hint: Rewrite so that you obtain $y=\pm\frac{b}{a}x\sqrt{\text{something}}$]. These two lines are often called oblique asymptotes. 

Now apply what you have just done to sketch the hyperbola $\frac{x^2}{25}-\frac{y^2}{9}=1$ and give the location of the foci. 
\end{problem}

The next three problems will help you use the basic equations of a hyperbola, together with shifting and reflecting, to study all ellipses whose major axis is parallel to either the $x$- or $y$-axis. 

\begin{problem} 
\marginpar{
	\thomasee{See 11.6: 27-34}
	}%
For each hyperbola below, graph the hyperbola (include the box and asymptotes) and give the coordinates of the foci and vertexes. 
\begin{enumerate}
\item $\ds 16x^2-25y^2=400$ [Hint: Divide by 400.]
\item $\ds \frac{(x-1)^2}{5}-\frac{(y-2)^2}{9}=1$
\end{enumerate}
\end{problem}

\begin{problem}
For the hyperbola $x^2+2x-2y^2+8y=9$, sketch a graph (include the box and asymptotes) and give the coordinates of the foci and vertexes. 
\end{problem}

\begin{problem} 
\marginpar{
	\thomasee{See 11.6: 35-38}
	}%
Given an equation of each hyperbola described below, and provide a rough sketch.
\begin{enumerate}
\item The vertexes are at $(2\pm 3,1)$ and foci at $(2\pm 5, 1)$.
\item The vertexes are at $(-1,3\pm 2)$ and foci at $(-1, 3\pm 5)$.
\end{enumerate}
\end{problem}

\bmw{Ask me about the reflective properties of a hyperbola in class, if I have not told you already. In particular, we can discuss lasers and long range telescopes. The following problem provides the basis to these reflective properties and is optional.  If you wish to present it, let me know. I'll have you type it up prior to presenting in class.}

\begin{problem*}[Optional]
Consider the hyperbola $\frac{x^2}{a^2}-\frac{y^2}{b^2}=1$ with foci $F_1=(c,0)$ and $F_2=(-c,0)$. 
Let $Q=(x,y)$ be a point on the hyperbola. 
Let $T$ be the tangent line to the hyperbola at point $Q$. 
Show that the angle between $F_1Q$ and $T$ is the same as the angle between $F_2Q$ and $T$. This shows that if you begin a ray from a point in the plane and head towards $F_1$ (where the wall of the hyperbola lies between the start point and $F_1$), then when the ray hits the wall at $Q$, it reflects off the wall and heads straight towards the other focus $F_2$.
\end{problem*}

\note{Do I want a problem that has them decide which is which, or is it enough to make them do that below?  I'm going with below. --- Karl moved the following problems to a whole new chapter.}






%%% Local Variables: 
%%% mode: latex
%%% TeX-master: "215-problems"
%%% End: 

\clearpage

%This split out the parametric equations from Woodruff's original "conic sections" chapter. I have found that the students could use a quick primer on parametric equations to see more success later at Valpo.

%\wrapup
%
%This breaks up the polar coordinates, and cylidrical/spherical coordinate
%Chapter from Woodruff's original version. This is because Valpo covers 
%Polar coordinates before Calc III, but needs to learn the others. So
%polar moved to a review chapter. Basically I just moved things from 
%Woodruff's original Chap 3+4 to 'New at Valpo' and 'Review at Valpo' chapters.

\chapter{Parametric Equations}
\minitoc \mtcskip
\input{03b-Parametric_Equations_v3}
\wrapup
\clearpage

\chapter{Polar and New Coordinate Systems}
\minitoc \mtcskip
\input{04-New-Coordinates_v3}
\wrapup
\clearpage

\chapter{Functions}
\minitoc \mtcskip
\input{05-Functions_v3}
\wrapup
\clearpage

\chapter{Derivatives}
\minitoc \mtcskip
\input{06-Derivatives_v3}
\wrapup
\clearpage


\chapter{Motion}
\vspace{-1.25cm}
\minitoc \mtcskip
\input{07-Motion_v3}
\wrapup

\chapter{Line Integrals}
\minitoc \mtcskip
\input{08-Line-Integrals_v3}
\wrapup



\chapter{Optimization}
\minitoc \mtcskip
\input{09-Optimization_v3}
\wrapup


\chapter{Double Integrals}
\minitoc \mtcskip
\input{10-Double-Integrals_v3}
\wrapup

%\end{document}


\chapter{Surface Integrals}
\minitoc \mtcskip
\input{11-Surface-Integrals_v3}
\wrapup

\chapter{Triple Integrals}
\minitoc \mtcskip

\noindent 
This unit covers the following ideas. In preparation for the quiz and exam, make sure you have a review guide containing examples that explain and illustrate the following concepts.  
\begin{enumerate}
\item Explain how to setup and compute triple integrals, as well as how to interchange the bounds of integration. Use these ideas to find area and volume.
\item Explain how to change coordinate systems in integration, with an emphasis on cylindrical, and spherical coordinates. Explain what the Jacobian of a transformation is, and how to use it.
\item Use triple integrals to find physical quantities such as center of mass, radii of gyration, etc. for solid regions.
\item Explain how to use the Divergence theorem to compute the flux of a vector fields out of a closed surface.
\end{enumerate}

\newpage

%\stepcounter{unitday}
%\uday
%\normalsize

\section{Triple Integral Definition and Applications}

After completing this section you will:
\begin{itemize}
\item Be comfortable setting up triple integrals for a solid region (domain)
\item Be able to switch the order of integration for a triple integral
\end{itemize}


Consider the iterated integral $$\ds \int_{-3}^3 \int_0^{\sqrt{9-y^2}}\int_0^{9-x^2-y^2} 1dzdxdy.$$ This is an integral of the form $\iiint_D dV$, which means along some solid region $D$ in the plane, we are adding up little bits of volume. This integral will give the volume of a very familiar solid region in space. Not every triple integral will correspond to an easily drawn or visualized (even with a computer) space. However, this one does.
\begin{problem}
Before we compute this integral, let's visualize it. The parenthesis have been added to help you parse this first example.
$$\ds \int_{-3}^3 (\int_0^{\sqrt{9-y^2}} (\int_0^{9-x^2-y^2} 1dz) dx ) dy.$$
\begin{enumerate}
	\item State the bounds for each variable as inequalities.
	\item Sketch the region $D$ in space.  
	\item Compute the innermost integral, and compare the resulting/remaining double integral to the first exercise in the double integral unit. 
	\item Now evaluate the remaining integrals (though you might want to change coordinate systems before doing so).
\end{enumerate}
\end{problem}

When working with double integrals, there were two different ways to set up the bounds for our integrals, as $dA=dxdy=dydx$.  When working with triple integrals, there are six different ways to order the differentials, and therefore set up bounds for our integrals. $$dV=dxdydz = dxdzdy = dydxdz=dydzdx=dzdxdy=dzdydx$$ 


Consider again the iterated integral $$\ds \int_{-3}^3 \int_0^{\sqrt{9-y^2}}\int_0^{9-x^2-y^2} dzdxdy = \int_0^9\int_0^{\sqrt{9-z}}\int_{-\sqrt{9-x^2-z}}^{\sqrt{9-x^2-z}} dydxdz$$ These two integrals are equal, but required switching the order of the bounds. There are 4 other iterated integrals that are equal to these integral which can also be found by switching the order of the bounds. 

\begin{problem}
Set up the equivalent integrals using the differential orderings: $dydzdx$ and $dxdzdy$.  We'll look at the remaining 2 in class (though you're welcome to finish them and present them with your work). 
\end{problem}

\begin{problem}
Consider the iterated integral $$\int_{-1}^1\int_0^{1-x^2}\int_0^{y} dzdydx.$$
The bounds for this integral describe a region in space which satisfies the 3 inequalities $-1\leq x\leq 1$, $0\leq y\leq 1-x^2$, and $0\leq z\leq y$.
\begin{enumerate}
 \item Draw the solid domain $D$ in space described by the bounds of the iterated integral.
\end{enumerate} 
There are 5 other iterated integrals equivalent to this one. 
\begin{enumerate}[resume]
	\item Set up the integral that use the differential ordering: $dydxdz$ 
	\item Set up the integral that use the differential ordering: $dxdzdy$.  
\end{enumerate}
We'll create the other 3 in class (though you are welcome to include them as part of your presentation)
\end{problem}

%\newpage
%\Large
%\textbf{In-Class}\\
%\normalsize

\note{Need to include a review problem on sketching planes and finding their intercepts. Particularly with fractional equations}

\begin{problem}
 In each exercise below, you'll be given enough information to determine a solid domain $D$ in space. 
\begin{enumerate}[a)]
	\item Draw the solid $D$ and 
	\item then set up an iterated integral (pick any order you want) that would give the volume of $D$.  
\end{enumerate}
Note: You don't need to evaluate the integral, rather you just need to set them up.
\begin{enumerate}
 \item The region $D$ under the surface $z=y^2$, above the $xy$-plane, and bounded by the planes $y=-1$, $y=1$, $x=0$, and $x=4$.
 \item The region $D$ in the first octant that is bounded by the coordinate planes, the plane $y+z=2$, and the surface $x=4-y^2$.
 \item The pyramid $D$ in the first octant that is below the planes $\ds\frac{x}{3}+\frac{z}{2}=1$ and $\ds\frac{y}{5}+\frac{z}{2}=1$. [Hint: Don't let $z$ be the inside bound.]
 \item The region $D$ that is inside both right circular cylinders $x^2+z^2=1$ and $y^2+z^2=1$.
\end{enumerate}
\end{problem}

We can find average value, centroids, centers of mass, moments of inertia, and radii of gyration exactly as before. We just now need to integrate using three integrals, and replace $ds$, $dA$ or $d\sigma$, with $dV$.  
\begin{problem}
 Consider the triangular wedge $D$ that is in the first octant, bounded by the planes $\ds\frac{y}{7}+\frac{z}{5}=1$ and $x=12$. In the $yz$ plane, the wedge forms a triangle that passes through the points $(0,0,0)$,  $(0,7,0)$, and $(0,0,5)$.  [Hint: It may help to sketch a picture first]
\begin{enumerate}
	\item Set up integral formulas that would give the centroid $(\bar x,\bar y, \bar z)$ of $D$.  
	\item Actually compute the integrals for $\bar y$. 
	\item Now state $\bar x$ and $\bar z$ by using symmetry arguments.
\end{enumerate}
\end{problem}

\begin{problem}
 Consider the tetrahedron $D$ in the first octant that is underneath the plane that intersects the coordinate axes in the three point $(a,0,0)$, $(0,b,0)$ and $(0,0,c)$, where you can assume that $a,b,c>0$.
\begin{enumerate}[a)]
	\item An equation of an ellipse that passes through $(a,0)$ and $(0,b)$ is $\ds\frac{x^2}{a^2}+\frac{y^2}{b^2}=1$.  
	\item An equation of a line through these same two points is $\ds\frac{x}{a}+\frac{y}{b}=1$.  
	\item An equation of an ellipsoid through the three points $(a,0,0)$, $(0,b,0)$, and $(0,0,c)$ is $\ds\frac{x^2}{a^2}+\frac{y^2}{b^2}+\frac{z^2}{c^2}=1$.
\end{enumerate}
\begin{enumerate}
  \item  Guess an equation of the plane through these same three points, and then verify that your guess is correct by plugging the 3 points into your equation. This will provide you with an extremely fast way to get an equation of a plane.
  \item Set up an iterated integral that would give the volume of $D$.
  \item If the density is $\delta(x,y,z) = 3x+2yz$, set up iterated integrals that would give the mass $m$\bmw{ and moment of inertia $I_y$ about the $y$-axis}.
\end{enumerate}
\end{problem}

%
%\uday
%\normalsize


\section{Changing Coordinate Systems: The Jacobian}
After finishing this section, you should...
\begin{itemize}
\item Be able to change between standard coordinate systems for triple integrals:
\begin{itemize}
\item Spherical Coordinates
\item Cylindrical Coordinates
\end{itemize}
\end{itemize}

Just as we did with polar coordinates in two dimensions, we can compute a Jacobian for any change of coordinates in three dimensions.  We will focus on cylindrical and spherical coordinate systems. \\

Remember that the Jacobian of a transformation is found by first taking the derivative of the transformation, then finding the determinant, and finally computing the absolute value.\\

\begin{problem}
 The cylindrical change of coordinates is: 
\begin{align*}
	x&=r\cos\theta, y=r\sin\theta, z=z\\
	\text{or in vector form}&\  \\
	\vec C(r,\theta,z) &= (r\cos\theta,r\sin\theta, z)
\end{align*}  
The spherical change of coordinates is: 
\begin{align*}
	x&=\rho\sin\phi\cos\theta,\ y=\rho\sin\phi\sin\theta,\ z=\rho\cos\phi\\
	\text{or in vector form}&\ \\
	\vec S(\rho,\phi,\theta) &= (\rho\sin\phi\cos\theta,\rho\sin\phi\sin\theta,\rho\cos\phi). 
\end{align*}

\begin{enumerate}
 \item Verify that the Jacobian of the cylindrical transformation is $\ds\frac{\partial(x,y,z)}{\partial(r,\theta,z)} = |r|$.  
	\begin{itemize}
		\item If you want to make sure you don't have to use absolute values, what must you require?
	\end{itemize}
 \item Verify that the Jacobian of the spherical transformation is $\ds\frac{\partial(x,y,z)}{\partial(\rho,\phi,\theta)} = |\rho^2\sin\phi|$.  
	\begin{itemize}
		\item If you want to make sure you don't have to use absolute values, what must you require?
	\end{itemize}
\end{enumerate}
\end{problem}

The previous exercise shows us that, provided we require $r\geq0$ and $0\leq \phi\leq \pi$, we can write:
$$dV=dxdydz = rdrd\theta dz = \rho^2\sin\phi d\rho d\phi d\theta,$$
\vskip0.1in
\textbf{Cylindrical coordinates} are extremely useful for problems which involve:
\begin{itemize}\itemsep2pt
\item cylinders
\item paraboloids
\item cones
\end{itemize}

\textbf{Spherical coordinates} are extremely useful for problems which involve:
\begin{itemize}\itemsep2pt
\item cones
\item spheres
\end{itemize}

%\newpage
\vskip0.1in

\begin{problem}\marginpar{See \href{http://aleph.sagemath.org/?z=eJytjkEOgjAQRfecojvaMBKsYIxhOEpNFaNNgDZItO3pRaguSExcuJo_8_Jf5i57Gltw4GMW9eeL0h02sj3WkkzXPfElWsETJziRXU18hdkcBP-Q0OSLaqAV5lMjrCWut5FB1ZpGndRwMI0eNjUNFEcK1MKqgIIBde_gIYOcQXhxHqCNHA0Os3THIpN8d-Z_U_qXLnh_lfKFNb1d9YOyJ0VbdCM}{Sage}.}%
 The double cone $z^2=x^2+y^2$ has two halves.  Each half is called a nappe. Set up an integral in the coordinate system of your choice that would give the volume of the region that is between the $xy$ plane and the upper nappe of the double cone $z^2=x^2+y^2$, and between the cylinders $x^2+y^2=4$ and $x^2+y^2=16$.  Then evaluate the integral.
\end{problem}

\begin{problem}
 Set up an integral in the coordinate system of your choice that would give the volume of the solid ball that is inside the sphere $a^2=x^2+y^2+z^2$. Compute the integral to give a formula for the volume of a sphere of radius $a$.  \bmw{Then set up (don't evaluate) an iterated integral that would give the moment of inertia $I_x$ about the $x$-axis, if the density is a constant, so $\delta =c$.}
\end{problem}


\begin{problem}\marginpar{See \href{http://aleph.sagemath.org/?z=eJxty00OgyAQQOF9L-IQB6PSnxUnMWqImpTEBgqTFji900V35m3e5vuYAFXCjKUSF6_ty-92sTT73ZFaoWgd34EgTX2dp14gJJQKFU_-T8EWr4J1fcbvkq1ke0675oaPH27i033BRL8tNAdD1umhQ24UB9CiL84}{Sage}.}%
Find the volume of the solid domain $D$ in space which is above the cone $z=\sqrt{x^2+y^2}$ and below the paraboloid $z=6-x^2-y^2$. Use cylindrical coordinates to set up and then evaluate your integral.  \newline [Hint: You'll need to find where the surface intersect, as their intersection will help you determine the appropriate bounds.]
\end{problem}

For the next several exercises be sure to check that you've correctly swapped bounds by having Sage or WolframAlpha actually compute all of the integrals.

\begin{problem}
Consider the region $D$ in space that is inside both the sphere $x^2+y^2+z^2=9$ and the cylinder $x^2+y^2=4$.\newline [Hint: Start by drawing the region.]
\begin{enumerate}
 \item Set up an iterated integral in Cartesian (rectangular) coordinates that would give the volume of $D$. 
 \item Set up an iterated integral in cylindrical coordinates that would give the volume of $D$. 
\end{enumerate}
\end{problem}

\begin{problem}
Consider the region $D$ in space that is both inside the sphere $x^2+y^2+z^2=9$ and yet outside the cylinder $x^2+y^2=4$. \newline [Hint: Start by drawing the region.]
\begin{enumerate}
 \item Set up two iterated integrals in cylindrical coordinates that would give the volume of $D$. 
	\begin{enumerate}[a)]
		\item For the first integral use the order $dzdrd\theta$.  
		\item For the second, use the order $d\theta dr dz$.
	\end{enumerate}
 \item Set up an iterated integral in spherical coordinates that would give the volume of $D$. 
\end{enumerate}
\end{problem}



\begin{problem}
The integral $\ds\int_{0}^{\pi}\int_{0}^{1}\int_{\sqrt{3}r}^{\sqrt{4-r^2}}rdzdrd\theta$ represents the volume of solid domain $D$ in space. Set up integrals in both rectangular coordinates and spherical coordinates that would give the volume of the exact same region.
\end{problem}

\begin{problem}
The temperature at each point in space of a solid occupying the region {$D$}, which is the upper portion of the ball of radius 4 centered at the origin, is given by $T(x,y,z) = \sin(xy+z)$.  Set up an iterated integral formula that would give the average temperature.   
\end{problem}



%\uday
%\normalsize

\section{The Divergence Theorem}
\instructor{Old day -3- split here}
\instructor{There are several more commented out exercises in the .tex file that could be included, but are not well edited/tested yet.}
After completing this section you should be able to...
\begin{itemize}
\item Use the Divergence Theorem to compute flux across a surface
\end{itemize}

In the double integral chapter, we learned a way to greatly simplify flux computations when working with simple closed curves.  Green's theorem stated that $\int_C \vec F\cdot \vec n\ ds = \iint_R (M_x+N_y) dA$.  The divergence of $\vec F$ is the quantity $\text{div}(\vec F) = M_x+N_y$. We saw this in definition \ref{definition of flux density in 2D} on page \pageref{definition of flux density in 2D}, when we defined the divergence, or flux density, of a vector field $\vec F$ at a point $P$ to be the flux per unit area. We then stated that $\text{div}(\vec F)=M_x+N_y$. We more generally defined the divergence in the previous unit. This generalizes the simplification to higher dimensions, and is called the divergence theorem. We'll repeat the definition of divergence, with its theorem, for reference.

\begin{theorem}[Divergence Theorem]
Let $S$ be a closed surface whose interior is the solid domain $D$. Let $\vec n$ be an outward pointing unit normal vector to $S$. Suppose that $\vec F(x,y,z)$ is a continuously differentiable vector field on some open region that contains $D$. Then the outward flux of $\vec F$ across $S$ can be computed by adding up, along the entire solid $D$, the flux per unit volume (divergence).  Symbolically, the divergence theorem states
$$\iint_S\vec F\cdot \vec n d\sigma =  \iiint_D \vec \nabla \cdot \vec F dV = \iiint_D \left(M_x+N_y+P_z\right) dV $$
for $S$ a closed surface with interior $D$ and outward normal $\vec n$.
\end{theorem}

%In 3D, the flux of $\vec F$ across $S$, $\iint_S\vec F\cdot \vec n d\sigma$, is a measure of flow across $S$ where $\vec n$ is a continuous unit normal vector to $S$.  Flux density at $(x,y,z)$ is found by creating a sphere $S_a$ of radius $a$ centered at $(x,y,z)$ with interior volume $V_a$ and outward normal vector $\vec n$, and considering the quotient of flux per volume given by $\frac{1}{V_a}\iint_{S_a} \vec F \cdot \vec n d\sigma$. By computing $\ds \lim_{a\to 0}\frac{1}{V_a}\iint_{S_a} \vec F \cdot \vec n d\sigma$, we obtain the divergence of $\vec F$ at $(x,y,z)$, also called the flux density. In a future mathematics course, we could prove that the divergence equals
%\begin{align*}
%\text{div}\vec F(x,y,z) 
%&= \vec \nabla\cdot \vec F 
%= \left(\frac{\partial }{\partial x},\frac{\partial }{\partial y},\frac{\partial }{\partial z} \right)\cdot (M,N,P) \\
%&= \frac{\partial M}{\partial x}+\frac{\partial N}{\partial y}+\frac{\partial P}{\partial z} 
%= M_x+N_y+P_z 
%.
%\end{align*}

\begin{problem}
 Consider the exact same vector field and box as exercise \ref{boxflux}.  We then have the vector field $\vec F=(x+y,y,z) $  and $S$ is the surface of the cube in the first quadrant bounded by {$ x=2,y=3,z=5 $}.
\begin{enumerate}
 \item Compute the divergence of $\vec F$. I.E find $\text{div}(\vec F) = M_x+N_y+P_z$.
\end{enumerate}
The divergence theorem states that if $S$ is a closed surface (has an inside and an outside), and the inside of the surface is the solid domain $D$, then the flux of $\vec F$ outward across $S$ equals the triple integral
$$\iint_S\vec F\cdot \vec n\ d\sigma = \iint\int_D \text{div}(\vec F)dV.$$
\begin{enumerate}[resume]
	\item Use the divergence theorem to compute the flux of $\vec F$ across $S$.\newline [Hint: Just as the area is found by adding up little bits of area, which is what we mean by $A=\iint_R dA$, the volume is found by adding up little bits of volume.] 
\end{enumerate}
\end{problem}


%\begin{problem}
%In problem \ref{sphere surface area element}, we found 
%$$\vec n d\sigma = \vec r_\phi\times \vec r_\theta d\phi d\theta = a^2\sin \phi (\sin\phi\cos\theta,\sin\phi\sin\theta,\cos\phi)d\phi d\theta$$ for a sphere of radius $a$.  
%Use this to compute the outward flux of $$\ds \vec F=\frac{\left<-x,-y,-z\right>}{(x^2+y^2+z^2)^{3/2}} $$ across a sphere of radius $a$. You should get a negative number since the vector field has all arrows pointing in. [Hint: Remember that for a sphere of radius $a$ we have $a^2=x^2+y^2+z^2$. When you perform the dot product of $\vec F$ and $\vec n$, you'll save yourself a lot of time if you remember that $\vec u\cdot \vec u = |\vec u|^2$; the dot product of a vector with itself is the length squared.]
%
%\end{problem}
%
%
%\begin{problem}
%Repeat the previous problem, but this time don't use the formula from problem \ref{sphere surface area element}. In fact, you don't even need to parametrize the surface. Instead, if you are at the point $(x,y,z)$ on a sphere of radius $a$, give a formula for the outward pointing unit normal vector $\vec n$. Give this formula by only using a geometric argument.  Then find the outward flux of {$\ds \vec F=\frac{\left<-x,-y,-z\right>}{(x^2+y^2+z^2)^{3/2}} $} across a sphere of radius $a$. You should find that $\vec F\cdot \vec n$ simplifies to a constant, so that you never actually have to compute $d\sigma$. Then you can use known facts about the surface area of a sphere.
%
%\end{problem}


%In 3D, the flux of $\vec F$ across $S$, $\iint_S\vec F\cdot \vec n d\sigma$, is a measure of flow across $S$ where $\vec n$ is a continuous unit normal vector to $S$.  Flux density at $(x,y,z)$ is found by creating a sphere $S_a$ of radius $a$ centered at $(x,y,z)$ with interior volume $V_a$ and outward normal vector $\vec n$, and considering the quotient of flux per volume given by $\frac{1}{V_a}\iint_{S_a} \vec F \cdot \vec n d\sigma$. By computing $\ds \lim_{a\to 0}\frac{1}{V_a}\iint_{S_a} \vec F \cdot \vec n d\sigma$, we obtain the divergence of $\vec F$ at $(x,y,z)$, also called the flux density. In a future mathematics course, we could prove that the divergence equals
%\begin{align*}
%\text{div}\vec F(x,y,z) 
%&= \vec \nabla\cdot \vec F 
%= \left(\frac{\partial }{\partial x},\frac{\partial }{\partial y},\frac{\partial }{\partial z} \right)\cdot (M,N,P) \\
%&= \frac{\partial M}{\partial x}+\frac{\partial N}{\partial y}+\frac{\partial P}{\partial z} 
%= M_x+N_y+P_z 
%.
%\end{align*}

\note{I tried problems like this during the semester in the double integral section, and they didn't go very well.  Perhaps it was placement.  It was a waste of 20 minutes in the previous section.}

%\begin{problem}
%Compute the flux density at $(0,0,0)$ for the vector field $\vec F = \left<x,y,z\right>$. A sphere of radius $a$ has unit normal vector $\vec n = \frac{\left<x,y,z\right>}{|\left<x,y,z\right>|}$, so the flux is $\iint_S \left<x,y,z\right>\cdot \frac{\left<x,y,z\right>}{|\left<x,y,z\right>|}d\sigma 
%= \iint_S \sqrt{x^2+y^2+z^2}d\sigma 
%= a \iint_S d\sigma   = a4\pi a^2$, since the surface area of a sphere is $4\pi a^2$. The volume inside a sphere of radius $a$ is $\frac 43\pi a^3$. Hence $\lim_{a\to 0}\frac{3}{4\pi a^3}4\pi a^3 = 3$, which equals $\text{div}\vec F = M_x+N_y+P_z = 1+1+1=3$. 
%\end{problem}


%\begin{theorem}[Divergence Theorem]
%Let $S$ be a closed surface whose interior is the solid domain $D$. Let $\vec n$ be an outward pointing unit normal vector to $S$. Suppose that $\vec F(x,y,z)$ is a continuously differentiable vector field on some open region that contains $D$. Then the outward flux of $\vec F$ across $S$ can be computed by adding up, along the entire solid $D$, the flux per unit volume (divergence).  Symbolically, the divergence theorem states
%$$\iint_S\vec F\cdot \vec n d\sigma =  \iiint_D \vec \nabla \cdot \vec F dV = \iiint_D \left(M_x+N_y+P_z\right) dV $$
%for $S$ a closed surface with interior $D$ and outward normal $\vec n$.
%\end{theorem}


\begin{problem}
Let $S$ be the surface of the wedge in the first octant bounded by the planes $x=1$ and $\ds\frac{y}{2}+\frac{z}{3}=1$. Let $\vec F$ be the vector field $\left<x+3y^2,y^2-4x,2z+xy\right>$. Use the divergence theorem to compute the outward flux of $\vec F$ across $S$. Make sure you draw the wedge (you may find centroids and volume help complete this exercise rapidly).  
\end{problem}

\begin{problem}
Consider the vector field $\vec F = \left<yz,-xz,3xz\right>$.  Let $D$ be the solid region in space inside the cylinder of radius 4, above the plane $z=0$, and below the paraboloid $z=x^2+y^2$.  The surface $S$ consists of 3 portions, so computing the flux would require a rather time consuming process of parameterizing these 3 surfaces.  Instead, use the divergence theorem to compute the outward flux of $\vec F$ across the surface $S$.
\end{problem}

%%% Local Variables: 
%%% mode: latex
%%% TeX-master: "215-problems"
%%% End: 

%%This moves the limits from an earlier chapter to their
%%own chapter for coverage at the end of the semester, if 
%%there's time. 
%\chapter{Limits}
%\minitoc \mtcskip
%This extra unit returns to the idea of Limits.


\section{Limits}
In the previous chapter, we learned how to describe lots of different functions. In first-semester calculus, after reviewing functions, you learned how to compute limits of functions, and then used those ideas to develop the derivative of a function. The exact same process is used to develop calculus in high dimensions. One glitch that will prevent us from developing calculus this way in high dimensions is the epsilon-delta definition of a limit.  We'll review it briefly.  Those of you who want to pursue further mathematical study will spend much more time on this topic in future courses. 

In first-semester calculus, you learned how to compute limits of functions. Here's the formal epsilon-delta definition of a limit. 
\begin{definition}
 Let $f:\R\to\R$ be a function.
 We write $\ds \lim_{x\to c} f(x)=L$ if and only if for every $\epsilon>0$, there exists a $\delta>0$ such that $0<|x-c|<\delta$ implies $|f(x)-L|<\epsilon$.
\end{definition}
 We're looking at this formal definition here because we can compare it with the formal definition of limits in higher dimensions. The only difference is that we just put vector symbols above the input $x$ and the output $f(x)$.
\begin{definition}
 Let $\vec f:\R^n\to\R^m$ be a function.
 We write $\ds \lim_{\vec x\to \vec c} \vec f(\vec x)=\vec L$ if and only if for every $\epsilon>0$, there exists a $\delta>0$ such that $0<|\vec x-\vec c|<\delta$ implies $|\vec f(\vec x)-\vec L|<\epsilon$.
\end{definition}
We'll find that throughout this course, the key difference between first-semester calculus and multivariate calculus is that we replace the input $x$ and output $y$ of functions with the vectors $\vec x$ and $\vec y$. 
 
\begin{problem}
 For the function $f(x,y)=z$, we can write $f$ in the vector notation $\vec y=\vec f(\vec x)$ if we let $\vec x=(x,y)$ and $\vec y=(z)$. Notice that $\vec x$ is a vector of inputs, and $\vec y$ is a vector of outputs. 
 For each of the functions below, state what $\vec x$ and $\vec y$ should be so that the function can be written in the form $\vec y = \vec f (\vec x)$. \marginpar{The point to this problem is to help you learn to recognize the dimensions of the domain and codomain of the function.  If we write $\vec f:\R^n\to \R^m$, then $\vec x$ is a vector in $\R^n$ with $n$ components, and $\vec y$ is a vector in $\R^m$ with $m$ components.}  
\begin{enumerate}
 \item $f(x,y,z)=w$
 \item $\vec r(t)=(x,y,z)$
 \item $\vec r(u,v)=(x,y,z)$
 \item $\vec F(x,y)=(M,N)$
 \item $\vec F(\rho,\phi,\theta)=(x,y,z)$
\end{enumerate}
\end{problem}


You learned to work with limits in first-semester calculus without needing the formal definitions above. Many of those techniques apply in higher dimensions. 
The following problem has you review some of these technique, and apply them in higher dimensions.
\begin{problem}\marginpar{\thomasee{See 14.2: 1-30 for more practice.} \stewarts{See 14.2:5-8 for more practice.}}%
 Do these problems without using L'Hopital's rule.
%, as there is not a good substitute for L'Hopital's rule in higher dimensions. \note{check this.}
\begin{enumerate}
 \item Compute $\ds \lim_{x\to 2} x^2-3x+5$ and then $\ds\lim_{(x,y)\to (2,1)} 9-x^2-y^2$.
 \item Compute $\ds\lim_{x\to 3}\frac{x^2-9}{x-3}$ and then $\ds\lim_{(x,y)\to (4,4)} \frac{x-y}{x^2-y^2}$.
 \item Explain why $\ds\lim_{x\to 0}\frac{x}{|x|}$ does not exist. [Hint: graph the function.]
\end{enumerate}
\end{problem}



In first semester calculus, we can show that a limit does or does not exist by considering what happens from the left, and comparing it to what happens on the right.  You probably used the following theorem extensively. 
\begin{quote}
 If $y=f(x)$ is a function defined on some open interval containing $c$, then $\ds\lim_{x\to c}f(x)$ exists if and only if  $\ds\lim_{x\to c^-}f(x) = \ds\lim_{x\to c^+}f(x)$.
\end{quote}
 A limit exists precisely when the limits from every direction exists, and all directional limits are equal. In first-semester calculus, this required that you check two directions (left and right). This theorem generalizes to higher dimensions, but it becomes much more difficult to apply. 

\begin{example}
 Consider the function $\ds f(x,y)=\frac{x^2-y^2}{x^2+y^2}$.
Our goal is to determine if the function has a limit at the origin $(0,0)$. We can approach the origin along many different lines.

One line through the origin is the line $y=2x$. If we stay on this line, then we can replace each $y$ with $2x$ and then compute
$$\ds\lim_{\text{\footnotesize $\begin{array}{c}(x,y)\to(0,0)\\ y=0\end{array}$}}\frac{x^2-y^2}{x^2+y^2} 
= \lim_{x\to 0} \frac{x^2-(2x)^2}{x^2+(2x)^2}
= \lim_{x\to 0} \frac{-3x^2}{5x^2}
= \lim_{x\to 0} \frac{-3}{5}
=\frac{-3}{5}.$$
This means that if we approach the origin along the line $y=2x$, we will have a height of $-3/5$ when we arrive at the origin.
\end{example}
If the function $\ds f(x,y)=\frac{x^2-y^2}{x^2+y^2}$ has a limit at the origin, the previous example suggests that limit will be $-3/5$.
\begin{problem}
 Please read the previous example. Recall that we are looking for the limit of the function $\ds f(x,y)=\frac{x^2-y^2}{x^2+y^2}$ at the origin (0,0). 
\marginpar{You may want to look at a graph in 
\href{http://aleph.sagemath.org/?z=eJxL06jQqdS01aiIM9KtjDPS1AextEEsroKc_BLjFI00HaASXWMdY00djUoIQxMAoucONQ}{Sage}
or \href{http://wolfr.am/ioCqzX}{Wolfram Alpha} (try using the ``contour lines'' option). %http://www.wolframalpha.com/input/?i=plot+%28x%5E2-y%5E2%29%2F%28x%5E2%2By%5E2%29
 As you compute each limit, make sure you understand what that limit means in the graph.}
Our goal is to determine if the function has a limit at the origin $(0,0)$.
\begin{enumerate}
 \item In the $xy$-plane, how many lines pass through the origin $(0,0)$? Give an equation a line other than $y=2x$ that passes through the origin.  Then compute $$\ds\lim_{\text{\footnotesize $\begin{array}{c}(x,y)\to(0,0)\\ \text{your line}\end{array}$}}\frac{x^2-y^2}{x^2+y^2}
= \lim_{x\to 0} \frac{x^2-(?)^2}{x^2+(?)^2}=\ldots.$$
 \item Give another equation a line that passes through the origin.  Then compute $$\ds\lim_{\text{\footnotesize $\begin{array}{c}(x,y)\to(0,0)\\ \text{your line}\end{array}$}}\frac{x^2-y^2}{x^2+y^2}.$$
 \item Does this function have a limit at $(0,0)$? Explain. \marginpar{\thomasee{See 14.2: 41-50 for more practice.}\larsonfive{See Larson 13.2:23--36 and example 4 for more practice.} \stewarts{See 14.2: 9-12}}%
\end{enumerate}
\end{problem}


The theorem from first-semester calculus generalizes as follows.
\begin{quote}
 If $\vec y=\vec f(\vec x)$ is a function defined on some open region containing $\vec c$, then $\ds\lim_{\vec x\to \vec c}\vec f(\vec x)$ exists if and only if the limit exists along every possible approach to $\vec c$ and all these limits are equal.
\end{quote}
There's a fundamental problem with using this theorem to check if a limit exists. Once the domain is 2-dimensional or higher, there are infinitely many ways to approach a point. There is no longer just a left and right side. To prove a limit exists, you must check infinitely many cases --- that takes a really long time.  The real power to this theorem is it allows to show that a limit does not exist.  All we have to do is find two approaches with different limits.



\begin{problem}
\marginpar{See \href{http://aleph.sagemath.org/?z=eJyVVF1r2zAUfc-vuKQFy7PS2QndoCBY2dtgMFjfShtubLnW6lhCUlqrv37XlvOxtdtYEoKlc3TOufcKn50BfNFNBzcWn5Tj8FU5N_yMUfBZt618kBy-G6u6B1jmRTGbPaFlSc9Dks4-qc5Li6WfVbKGNauF6szOrze6Z7SDu9YL1r8L6XvW3y-zcL9MUz6D8WP8W-Sc5ymHFjeyFfNvmvSv4JyRWyrO5xyeVeUbUeQHFTTGaiyb4g2xRX9Qup5oUJBc-LvU8g0pSv9aa_laa5JKr8aHPuchF8aPi6GFHhq_bVlyDW5naywl7eoNbtoADTpAaNVWeUAPviFsKB9UPSysBOWg0wTS2aqSHZQNdg-0TU-61TayLpJ0MIv2diDQAHifLwr6y4oYMExAoHgEhAOwTyVIsvN6Z9em1Z7VPErx6SQvt2hE8kP6hI_eG7Tixu4IiMecuJx6MRa0jpWIWBFjY1_SeFQkVlYJd-pFilXOX7StpBWreLqmsnCokB3mzI9zmrp8EjwTamtaVQ6WQ_CwwH30ffLouWmxfEymohv9zE5yZpNYRLGXTqC1xGFR6ra44-M13a-XcZ1mEy2fzEYi3ebjxsA85eV8UURCzov0aJgJL3u_qti8p9t19Mkuislqj4f5r_oj4wR_me_lCZkcjBiaQ2j9W3NAG6TeBZFffOADJzbEiY-XFJpy_d9QTCYMWtxKb1UZB0J3EXnNgkB6EcDhVm1VJ449ozX24tgy36jysZPOidUf52f-qYLOyNKvLXqlxW3B6XuXzn4C8NyNiQ}{Sage}.}%
\marginpar{\thomasee{See 14.2: 41-50 for more practice.}\larsonfive{See Larson 13.2:9--36 for more practice.}\stewarts{See 14.2: 13-22 for a mix of functions with limits and without}}%
 Consider the function $\ds f(x,y) = \frac{xy}{x^2+y^2}$.  Does this function have a limit at $(0,0)$?  Examine the function at $(0,0)$ by considering the limit as you approach the origin along several lines. 
\end{problem}

In all the examples above, we considered approaching a point by traveling along a line. Even if a function has a consistent limit along EVERY line, that is not enough to guarantee the function has a limit. The theorem requires EVERY approach, which includes parabolic approaches, spiraling approaches, and more. For our purposes, checking along straight lines will do.  If you are interested in seeing an example of a function $f(x,y)$ so that the limit at $(0,0)$ along every straight line $y=mx$ exists and equals 0, but the function has no limit at $(0,0)$, then please ask.  Alternately, if this interests you, try coming up with an example yourself, and then come show me when you get it. This is a fun challenge.

\begin{problem*}[Challenge]
 Give an example of a function $f(x,y)$ so that the limit at $(0,0)$ along every straight line $y=mx$ exists and equals 0.  However, show that the function has no limit at $(0,0)$ by considering an approach that is not a straight line.
\end{problem*}
%\wrapup

\end{document}
